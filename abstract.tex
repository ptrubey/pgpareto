
We consider a constructive definition of the multivariate Pareto that factorizes the 
random vector into a radial component and an independent angular component; The former
following a univariate Pareto distribution, and the latter defined on the surface of the 
positive orthant of the unit hypercube.  In this paper, we  propose a method for
inferring the distribution of the angular component.  We identify its support 
as the limit of the positive orthants of the unit $p$--norm spheres.
To this effect, we introduce a projected gamma family of
distributions defined as the projection of a vector of independent gamma random variables
onto the $p$--norm sphere.  This family serves as a building block for a flexible family
of distributions obtained as a Dirichlet process mixture of projected gammas.  For 
model assessment and comparison, we discuss model scoring methods appropriate 
to distributions on the unit hypercube.  In particular, working with the energy score
criterion, we develop a kernel metric appropriate to the hypercube that produces a 
proper scoring rule.  We then present a simulation study to compare 
different modeling choices using the proposed scoring rules.  Finally we apply our 
approach to describe the dependence structure of the extreme values of the magnitude 
of the integrated vapor  transport (IVT), a variable that describes the rate of flow 
of moisture in the atmosphere  along the coast of California for the years of 1979 
through 2020.  We find a clear but heterogeneous geographical dependence.


% EOF