\documentclass[12pt]{article}
\usepackage[margin=1in]{geometry}
\usepackage{amsmath}
\usepackage{amssymb}
\usepackage{bm}
% \usepackage{setspace}
% \linespread{1.8}

\begin{document}
\paragraph{DP mixture of projected Gammas}
The DP mixture of projected Gammas model uses projected Gamma as a kernel density.
\begin{equation}
    \begin{aligned}
        \bm{y}_i\mid\bm{\alpha}_i,\bm{\beta}_i &\sim \mathcal{PG}(\bm{y}_i\mid\bm{\alpha}_i,\bm{\beta}_i)\\
        \bm{\alpha}_i,\bm{\beta}_i &\sim G\\
        G &\sim \mathcal{DP}\left(\eta,G_0\right)
    \end{aligned}
\end{equation}
Let $n_j^{-i}$ be the number of observations in cluster $j$ not including observation $i$.  Let $J^{-i}$ be the number of extant observations, not including any singletons that observation $i$ are in. 
Under this model, the probability of cluster membership is proportional to
\begin{equation*}
    \text{P}\left[\delta_i = j\mid\ldots\right] \propto \begin{cases}
        n_j^{-i}\mathcal{PG}\left(\bm{y}_i\mid\bm{\alpha}_j,\bm{\beta}_j\right)\hspace{0.5cm} &\text{~for~}j \in \lbrace 1,\ldots,J^{-i}\rbrace\\
        \eta\int\mathcal{PG}\left(\bm{y}_i\mid\bm{\alpha}_j,\bm{\beta}_j\right)dG_0(\bm{\alpha}_j,\bm{\beta}_j)\hspace{0.5cm} &\text{~for~}j = J^{-i} + 1.
        \end{cases}
\end{equation*}
Of course, if $G_0$ is not a conjugate prior for the kernel distribution, then the integral is difficult.  Instead, using algorithm 8 from Neal~(2000), we draw $m$ candidate clusters, $\bm{\alpha}_j,\bm{\beta}_j$ for $j = J^{-i} + 1,\ldots, J^{-i} + m$ from $G_0$. Then, the probability of cluster membership is proportional to
\begin{equation}
    \text{P}\left[\delta_i = j\mid\ldots\right] \propto \begin{cases}
        n_j^{-i}\mathcal{PG}\left(\bm{y}_i\mid\bm{\alpha}_j,\bm{\beta}_j\right)\hspace{0.5cm} &\text{~for~}j \in \lbrace 1,\ldots,J^{-i}\rbrace\\
        \frac{\eta}{m}\mathcal{PG}\left(\bm{y}_i\mid\bm{\alpha}_j,\bm{\beta}_j\right)\hspace{0.5cm} &\text{~for~}j \in \lbrace J^{-i} + 1,\ldots, J^{-i} + m\rbrace.
        \end{cases}
\end{equation}
Of course, if a candidate cluster is selected, then $\delta_j = J + 1$, and $J = J^{-i} + 1$.

A key feature of the the projected Gamma distribution is its computational properties.  With respect to the distributional parameters $\bm{\alpha}_i,\bm{\beta}_i$, the projected Gamma density is proportional to
\begin{equation*}
    \mathcal{PG}(\bm{y}\mid\bm{\alpha}_i,\bm{\beta}_i) \propto \int\prod_{\ell = 1}^d\mathcal{G}\left(ry_{i\ell}\mid\alpha_{i\ell},\beta_{o\ell}\right)r^{d-1}dr,
\end{equation*}
which implies that, by reintroducing the latent radial component $r_i$ via sampling from its full conditional,
\begin{equation}
    r_i\mid\bm{\alpha}_i,\bm{\beta}_i \sim \mathcal{G}\left(\sum_{\ell = 1}^d\alpha_{i\ell}, \sum_{\ell = 1}^d\beta_{\ell} y_{i\ell}\right),
\end{equation}
the likelihood for $\bm{\alpha},\bm{\beta}$ becomes separable by dimension.  We stress this point: \emph{given} $r$, the distributional parameters $(\alpha_j,\beta_j)_{\ell}$ are independent in the likelihood.

The full conditional for $\bm{\alpha}_j,\bm{\beta}_j$ is then proportional to
\begin{equation}
    f(\bm{\alpha}_j,\bm{\beta}_j\mid \bm{Y},\bm{r},\bm{\delta},\ldots) \propto \prod_{i:\gamma_i = j}\prod_{\ell = 1}^d\mathcal{G}\left(r_iy_{i\ell}\mid\alpha_{j\ell},\beta_{j\ell}\right) \times dG_0(\alpha_{\ell},\beta_{\ell})
\end{equation}
The first form of the centering distribution we explore is a product of independent Gammas.
\begin{equation}
    G_0(\bm{\alpha}_j,\bm{\beta}_j\mid \bm{\xi},\bm{\tau},\bm{\zeta},\bm{\sigma}) = \prod_{\ell = 1}^d\mathcal{G}(\alpha_{j\ell}\mid \xi_{\ell},\tau_{\ell})\times\prod_{\ell = 2}^d\mathcal{G}(\beta_{j\ell}\mid\zeta_{\ell},\sigma_{\ell})
\end{equation}
This model is completed with Gamma priors on $\xi_{\ell}$, $\tau_{\ell}$, $\zeta_{\ell}$, $\sigma_{\ell}$.  Optionally, placing a Gamma prior on $\eta$ allows updating via the procedure outlined in Escobar \& West (1995).  In further analysis, we refer to this model as the \emph{projected Gamma--Gamma} (PG--G) model.  An advantage to using a Gamma centering distribution for the rate parameter is, as a conjugate prior, the rate parameters can easily be integrated out for inference on $\bm{\alpha}_j$.  After doing so, the full conditional for $\alpha_{j\ell}$ takes the form
\begin{equation}
    \pi(\alpha_{j\ell}\mid \bm{r},\bm{Y},\bm{\gamma},\xi_\ell,\tau_\ell,\zeta_\ell,\sigma_\ell) \propto \frac{\left(\prod_{i:\gamma_i = j}r_iy_{i\ell}\right)^{\alpha_{j\ell} - 1}\alpha_{j\ell}^{\xi_\ell - 1}e^{-\tau_\ell \alpha_{j\ell}}\times\Gamma\left(n_j\alpha_{j\ell} + \zeta_{\ell}\right)}{\Gamma^{n_j}(\alpha_{j\ell})\left(\sum_{i:\gamma_i = j}r_iy_{i\ell} + \sigma_{\ell}\right)^{n_j\alpha_{j\ell} + \zeta_{\ell}}}
\end{equation}
for $\ell = 2,\ldots,d$.  For $\ell = 1$, as $\beta_{1} := 1$, the full conditional takes the form
\begin{equation}
    \label{eqn:alpha1update}
    \pi(\alpha_{j1}\mid\bm{r},\bm{Y},\bm{\gamma},\xi_1,\tau_1) \propto \frac{\left(\prod_{i:\gamma_i = j}r_iy_{i1}\right)^{\alpha_{j1} - 1}\alpha_{j1}^{\zeta_1 - 1}e^{-\tau_1\alpha_{j1}}}{\Gamma^{n_j}(\alpha_{j1})}.
\end{equation}
Updating these parameters can be done via a simple Metropolis-within-Gibbs step.  In our analysis, we first transform $\alpha_{j\ell}$ to log scale, and use a normal proposal density.
The full conditional for $\beta$ is then
\begin{equation}
    \beta_{j\ell}\mid\bm{r},\bm{Y},\alpha,\zeta_{\ell},\sigma_{\ell} \sim \mathcal{G}\left(\beta_{j\ell}\mid n_j\alpha_{j\ell} + \zeta_\ell, \sum_{i:\gamma_i = j}r_iy_{i\ell} + \sigma_{\ell}\right),
\end{equation}
for $\ell = 2,\ldots, d$.  Updating $\beta_{j\ell}$ is done via a Gibbs step.  The hyper-parameters $\xi_{\ell},\tau_{\ell},\zeta_{\ell},\sigma_{\ell}$ follow similar Gamma-Gamma update relationships.  We also explore a restricted form of this model, where $\beta_{\ell} = 1$ for all $\ell$.  Under this model, we use the full conditional in Equation~\ref{eqn:alpha1update} for all $\ell$, and omit inference on $\bm{\zeta},\bm{\sigma}$.  We refer to this model as the \emph{projected restricted Gamma--Gamma} (PRG--G) model.

The second form of centering distribution we explore is a multivariate log-normal distribution on the shape parameters $\bm{\alpha}_j$, with independent Gamma $\beta_{j\ell}$ rate parameters.  
\begin{equation}
    G_0\left(\bm{\alpha}_j,\bm{\beta}_j\mid\bm{\mu},\bm{\Sigma},\zeta,\sigma\right) = \mathcal{LN}\left(\bm{\alpha}_j\mid\bm{\mu},\bm{\Sigma}\right)\times\prod_{\ell = 2}^d\mathcal{G}\left(\beta_{j\ell}\mid\zeta_{\ell},\sigma_{\ell}\right)
\end{equation}
This model is completed with a normal prior on $\bm{\mu}$, an inverse Wishart prior on $\bm{\Sigma}$, and Gamma priors on $\zeta_{\ell}$ and $\sigma_{\ell}$, and $\eta$.  This model is denoted as the \emph{projected Gamma--log-normal} (PG--LN) model.  We also explore a restricted Gamma form of this model as above, where $\beta_{\ell} = 1$ for all $\ell$.  This is denoted as the \emph{projected restricted Gamma--log-normal} (PRG--LN) model.  in the interest of brevity, we omit the inferential forms for this model, sufficing to say that updating $\bm{\alpha}_j$ is accomplished with a joint Metropolis-within-Gibbs step; the normal prior for $\mu$ is conjugate for the log-normal $\log\bm{\alpha}_j$, and updated via a Gibbs step; and an inverse Wishart prior for $\Sigma$ is again conjugate to the log-normal $\bm{\alpha}_j$, meaning it too can be updated via a Gibbs step.

\end{document}

% EOF
