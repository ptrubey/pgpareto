\section{Conclusion\label{sec:conclusion}}
In this paper, we have built upon the definition of the multivariate Pareto 
    described in \cite{ferreira2014} to establish a useful representation of 
    its dependence structure through the distribution of its angular component,
    which is supported on the positive orthant of the unit hypersphere under 
    the $\mathcal{L}_{\infty}$ norm, $\mathbb{S}_{\infty}^{d-1}$.
    Due to the inherent difficulty of 
    obtaining the likelihood of distributions with support on ${\mathbb S}^{d-1}_\infty$ 
    our method transforms the data to ${\mathbb S}_{p}^{d-1}$, fits then using mixtures of 
    products of independent gammas, then transforms the predictions back to 
    ${\mathbb S}^{d-1}_\infty$. As ${\mathbb S}_{p}^{d-1}$ converges to 
    ${\mathbb S}_{\infty}^{d-1}$ as  $p\to\infty$, we expect the proposed 
    resampling to be efficient for large enough $p$. In fact, our exploration 
    of the simulated and real data indicates that the procedure is robust to 
    the choice of moderately large values of $p$.
    Our method includes two inferential steps. The first consists of the
    estimation of the marginal Pareto distributions; the second consists 
    of the estimation of the angular density. Parameter uncertainty incurred in
    the former is not propagated to the latter. Conceptually, an integrated approach 
    that accounts for all the estimation uncertainty is conceivable. Unfortunately, 
    this leads to posterior distributions with complex data dependent restrictions 
    that are very difficult to explore, especially in large dimensional settings. In 
    fact, our attempts to fit a simple parametric model for the marginals and the
    angular measure jointly in several dimensions were not successful.

In this paper we have focused on a particular representation of the multivariate
    Pareto distribution for PoT inference on extreme values. To this end, our model 
    provides a computationally efficient and flexible approach. An interesting extension
    of the proposed model is to consider regressions of extreme value responses, due
    to extreme value inputs following the ideas in  \cite{carvalho2022}. This will 
    produce PoT based Bayesian non-parametric extreme value regression models. More 
    generally,  models that allow for covariate-dependent extremal dependence 
    \citep{mhalla2019} could be considered. In addition, we notice that our approach 
    is based on flexibly  modeling angular distributions for any $p$-norm. As such, 
    it can be applied to other problems focused on high dimensional directional 
    statistics  constrained  to a cone of directions. 

Developing an angular measure specifically in $\mathbb{S}_{\infty}^{d-1}$ provides 
    two benefits over $\mathbb{S}_p^{d-1}$.  First, the transformation to 
    $\mathbb{S}_{\infty}^{d-1}$ is unique.  Recall that 
    Equation~\eqref{eqn:pnormtinv} gives $y_d$ as a function
    of $y_1,\ldots,y_{d-1}$. An analogous expression can be obtained for
    any $y_{\ell}$. This indicates that there are $d$ equivalent transformations, 
    each yielding a different
    Jacobian and, for $p>1$, potentially resulting in a different density.  
    Second, evaluation of geodesic distances on $\mathbb{S}_p^{d-1}$ is not 
    straightforward.  However, we have demonstrated a computationally efficient 
    upper bound on geodesic distance on $\mathbb{S}_{\infty}^{d-1}$.  Accepting
    these foibles, it would be interesting to explore the distribution on 
    $\mathbb{S}_{p}^{d-1}$, 
 
The computations in this paper were performed on a desktop computer with an 
    AMD Ryzen 5000 series processor. The program is largely single-threaded, so 
    computation time is not dependent on available core count.  In each case, 
    we run the MCMC chain for 50\,000 iterations, with a burn-in of 40\,000 
    samples.  Fitting the PG-G model on the ERA5 dataset took approximately 
    15 minutes.  Work is in progress to optimize the code, and explore 
    parallelization where possible.  We are also exploring alternative 
    computational approaches that will make it feasible to tackle very high 
    dimensional problems, such as variational Bayes. In fact, to elaborate on 
    the study of IVT, there is a need to consider several hundreds, if not thousands, 
    of grid cells over the Pacific Ocean in order to obtain a good description of 
    atmospheric events responsible for large storm activity over California.  

% §6 calls for a view of the authors on what 
%     could be related exercises—or even extensions of this research and future 
%     work. Using the proposed Bayesian multivariate PoT construction for 
%     devising a regression model for an an extreme value response on an extreme 
%     value covariate is one example of a natural follow-up; see 
%     \cite{carvalho2022} for related ideas. Conditional versions of the angular 
%     measure (e.g. \cite{carvalho2016}, \cite{castro2018}, \cite{escobar2018}, 
%     \cite{mhalla2019}) would lead to a conditional multivariate 
%     peaks-over-threshold model, where dependence between exceedances could 
%     change along with a covariate. Also, perhaps the here proposed DPM of 
%     projected gammas—as well as predictor-dependent versions of it—could be of 
%     interest in other settings beyond angular measures?


% EOF
