\section{Conclusion\label{sec:conclusion}}
In this paper, we have built upon the definition of the multivariate Pareto described in \cite{ferreira2014}
  to establish a useful parametric representation of its dependence structure, through describing
  the distribution of its angular component; it being supported on the positive orthant of the unit
  hypersphere under the $\mathcal{L}_{\infty}$ norm, ${\mathbb S}_{\infty}^{d-1}$.  We then demonstrated an
  inherent difficulty in establishing a distribution supported in this space.  To build that representation,
  we presented a distribution based on a product of independent Gammas, projected onto a supporting manifold
  ${\mathbb S}_{p}^{d-1}$, which will converge to ${\mathbb S}_{\infty}^{d-1}$ as $p\to\infty$.  We then
  established a means of model comparison on ${\mathbb S}_{\infty}^{d-1}$ through the use of a negative
  definite kernel defined in that space.

We used integrated vapor transport data---a measure of water in the atmosphere---as a motivating example for
  our model.  This data subsets California into a series of grid cells, and provides daily IVT values for
  each cell.  We used our model to develop measures of pairwise extremal dependence between grid cells, which
  could be of interest to researchers, actuaries, hydrologists and even farmers.  We additionally developed
  conditional survival functions---$\text{P}\left[Z_{\ell} > z_{\ell}\mid Z_{\not\ell} > z_{\not\ell}\right]$---which
  provide a more complete understanding of the dependence structure.
