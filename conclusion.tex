\section{Conclusion\label{sec:conclusion}}
In this paper, we have built upon the definition of the multivariate Pareto described in \cite{ferreira2014}
  to establish a useful parametric representation of its dependence structure, through
  the distribution of its angular component, which is supported on the positive orthant of the unit
  hypersphere under the $\mathcal{L}_{\infty}$ norm.  Due to the an
  inherent difficulty of obtaining a distribution with support on ${\mathbb S}^{d-1}_\infty$
  we projected distributions based on products of independent gammas, 
  onto the manifold
  ${\mathbb S}_{p}^{d-1}$, and the projected samples of the resulting
  probability distribution onto ${\mathbb S}_{\infty}^{d-1}$.  
  As ${\mathbb S}_{p}^{d-1}$ converges to ${\mathbb S}_{\infty}^{d-1}$ as $
  \to\infty$ we expect the resampling to be efficient for large enough
  $p$. In fact, our exploration of the simulated and real data indicate
  that the procedure is robust to the choice of moderately large values
  of $p$. 
 
Our simulations explored simulated data from
    mixtures of projected Gammas with varying degrees of complexity---both in terms of dimensionality and 
    in number of mixture components.  From this, we learned that the additional flexibility 
    offered by varying the $\beta$ term---the rate parameter of the Gamma distribution---does not result 
    in increased model fidelity in a mixture setting.  Within the tested range, between 3 and 20 
    dimensions, we observed that the additional information that could be captured using the log--normal 
    centering distribution did not translate to additional model fidelity.  However, when we looked at 
    real data, with 47 dimensions, there was an effect.
    
The computations performed in this paper were performed on a desktop computer with an AMD Ryzen 5950X processor.
  The program is largely single-threaded, so computation time is not dependent on available core count.  In each
  case, we run the MCMC chain for 50\,000 iterations, with a burn-in of 20\,000 samples.  Fitting the PG--G model
  on the ERA5 dataset took approximately 30 minutes.  Work is in progress to optimize the code, and explore
  parallelization where possible.  We are also exploring alternative computational approaches that will make it
  feasible to tackle very high dimensional problems. In fact, to elaborate on the study of IVT, there is a need 
  to consider several hundreds, if not thousands, of grid cells over the Pacific Ocean in order to obtain a 
  good decsription of atmospheric events responsible for large storm activity over California.


% {\bf ( describe
% the computer)} and in each case we run the MCMC for {\bf XXX} iterations
% with a burn in of {\bf XXX} samples. For the large simulated data and the
% ERA5 dataset the process of fitting the model took {\bd YYY} minutes. Work is in progress to optimize the
% code to perform the MCMC {\bf (insert some comment here if you think it is appropriate)}. We are also 
% exploring alternative computational approaches that will make it feasible to tackle very high dimensional
% problems. In fact, to elaborate on the study of IVT, there is a need to consider several hundreds, if not
% thousands, of grid cells over the Pacific Ocean in order to obtain a good decsription of atmospheric events responsible for large storm activity over California.

%We used integrated vapor transport data---a measure of water in the atmosphere---from the ERA-Interim and ERA5
%    models as a motivating example for our model.  This data subsets California into a series of grid cells,
%    and provides daily IVT values for each cell.  For the ERA-Interim data, having determined from
%    Table~\ref{tab:dev} that the optimal model presented was PRG--G, we used this fitted model to develop
%    $\Chi$, a measure of pairwise extremal dependence between grid cells.  This provides a useful summary
%    statistic for describing the relationship between two sites.  Granted this relationship is pairwise, it
%    can not represent the full dependence structure; so recognizing these limits, we also presented 
%    conditional survival 
%    functions---$\text{P}\left[Z_{\ell} > z_{\ell}\mid Z_{\neg\ell} > z_{\neg\ell}\right]$---which
%    provide a more complete understanding of the dependence structure of $\bm{Z}$.
