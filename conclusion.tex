\section{Conclusion\label{sec:conclusion}}
In this paper, we have built upon the definition of the multivariate Pareto 
    described in \cite{ferreira2014} to establish a useful representation of 
    its dependence structure, through the distribution of its angular component,
    which is supported on the positive orthant of the unit hypersphere under 
    the $\mathcal{L}_{\infty}$ norm.  Due to the inherent difficulty of 
    obtaining a distribution with support on ${\mathbb S}^{d-1}_\infty$ our 
    method projects distributions, supported on the positive quadrant and 
    based on products of independent gammas,  onto the manifold 
    ${\mathbb S}_{p}^{d-1}$. Samples of the resulting probability distribution 
    are then projected onto ${\mathbb S}_{\infty}^{d-1}$.  As 
    ${\mathbb S}_{p}^{d-1}$ converges to ${\mathbb S}_{\infty}^{d-1}$ as 
    $p\to\infty$, we expect the resampling to be efficient for large enough
    $p$. In fact, our exploration of the simulated and real data indicate
    that the procedure is robust to the choice of moderately large values
    of $p$. 
 
We explored the behavior of the proposed model using simulated data from
    mixtures of projected Gammas with varying degrees of complexity---both 
    in terms of dimensionality and in number of mixture components.  From this,
    we observed that the additional flexibility offered by varying the $\beta$
    term---the rate parameter of the gamma distribution---can offer additional
    model fidelity.  This result is not borne out in our real data example,
    for which the simpler model is vastly preferred.  Additionally, we observe
    that the choice of the log-normal centering distribution and additional
    flexibility it offers did not translate to additional model fidelity.  This
    is borne out in the real data, where on the ERA--Interim dataset it performed
    only comparably, but on ERA--5 it performed significantly worse.
    
    % From this,
    % we learned that the additional flexibility offered by varying the $\beta$ 
    % term---the rate parameter of the Gamma distribution---does not result in 
    % increased model fidelity in a mixture setting.  Within the tested range, 
    % between 3 and 20 dimensions, we observed that the additional information 
    % provided by a log--normal centering distribution did not translate to 
    % additional model fidelity.  However, for real data, with 47 dimensions, 
    % we did observe an effect.
    
The computations in this paper were performed on a desktop computer with an 
    AMD Ryzen 5950X processor. The program is largely single-threaded, so 
    computation time is not dependent on available core count.  In each case, 
    we run the MCMC chain for 50\,000 iterations, with a burn-in of 40\,000 
    samples.  Fitting the PG--G model on the ERA5 dataset took approximately 
    15 minutes.  Work is in progress to optimize the code, and explore 
    parallelization where possible.  We are also exploring alternative 
    computational approaches that will make it feasible to tackle very high 
    dimensional problems. In fact, to elaborate on the study of IVT, there is 
    a need to consider several hundreds, if not thousands, of grid cells over
    the Pacific Ocean in order to obtain a good description of atmospheric 
    events responsible for large storm activity over California.  Moving beyond
    this model

% §6 calls for a view of the authors on what 
%     could be related exercises—or even extensions of this research and future 
%     work. Using the proposed Bayesian multivariate PoT construction for 
%     devising a regression model for an an extreme value response on an extreme 
%     value covariate is one example of a natural follow-up; see 
%     \cite{carvalho2022} for related ideas. Conditional versions of the angular 
%     measure (e.g. \cite{carvalho2016}, \cite{castro2018}, \cite{escobar2018}, 
%     \cite{mhalla2019}) would lead to a conditional multivariate 
%     peaks-over-threshold model, where dependence between exceedances could 
%     change along with a covariate. Also, perhaps the here proposed DPM of 
%     projected gammas—as well as predictor-dependent versions of it—could be of 
%     interest in other settings beyond angular measures?


% EOF
