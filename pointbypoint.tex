%\documentclass[sn-basic]{sn-jnl}
\documentclass[10pt]{article}
\usepackage{natbib}
\usepackage[margin=1in]{geometry}
\bibliographystyle{plainnat}

\usepackage{amsmath}
\usepackage{amssymb}
\usepackage{amsthm}
\usepackage{hyperref}
\usepackage{graphicx}
\usepackage{xcolor}
\usepackage{booktabs}
\usepackage{soul}
\usepackage{bm}
\usepackage{subcaption}
\usepackage{wrapfig}

\newcommand{\comment}[1]{\textcolor{teal}{#1}}
\newcommand{\bruno}[1]{\textcolor{red}{#1}}

\begin{document}

\section*{Review 1}
\begin{itemize}
\item \emph{Moment constraint}: As can be seen, for example, in \cite[Equation 4.1]{einmahl2009}
    there is a moment constraint on the angular measure for any $\mathcal{L}_p$ norm with 
    $p \in [1, \infty]$ (nb including $p = \infty$).  Hence, why is there no moment constraint in 
    the submitted 
    manuscript? Some papers on inference for the angular measure have not taken into consideration 
    that constraint (e.g. \cite{einmahl2001}), though most modern approaches do account for it. If 
    the moment constraint is being ignored here, I would be willing to overlook this provided that 
    this is openly acknowledged---as well as what are the consequences of this. Otherwise, this needs 
    to be further clarified.

    % \comment{Both reviewers comment on the moment constraint.  I forget if it's this one, or
    % the reference provided by the other reviewer, but I don't think they understood the reference
    % directly; they were using the $\mathcal{L}_1$ norm in $p$ dimensions, rather than the 
    % $\mathcal{L}_p$ norm as described in our paper.  A massive difference in meaning of notation.
    % I'll find the reference again and confirm.}
    
We thank the referee for calling our attention to
    the issue of moment restrictions. The approach considered in the
    paper follows that of \cite{rootzen2018}, consisting on modeling the
    distribution of the extreme observations, conditional on them being
    above a threshold, say $H$.  The approach considered in
    \cite{EiSe2009} focuses in the limiting joint exponent measure. More
    precisely, for the two dimensional case, they assume that
    \[
        sPr(s^{-1}(Z_1,Z_2) \in A) \rightarrow \mu(A),
    \]
    where $\mu$ is the exponent measure. The moment conditions are
    the result of assuming that both marginals of $\mu$ correspond to a
    standard Pareto distribution. Such assumption is not compatible
    with the conditional distribution approach considered in our paper.
    In fact, as noted in \cite{KiRoSeWa2019}, typically the margins of
    $H$ are not univariate Pareto, due to the difference between the 
    conditioning events in the joint and the marginal cases. Thus, no
    moment conditions arise in our model. This has now been clarified in the paper.

We note in passing that the moment conditions for the spectral distribution, say 
    $\Phi$, in the two-dimensional case, using the infinity norm, imply that 
    \[
        \int_0^{\pi/4} \Phi(d\theta) = \int_{\pi/4}^{\pi/2}
    \Phi(d\theta).
    \]
    The former looks like a fairly strong symmetry restriction that is unlike to 
    be realistic in practice. Thus, imposing moment conditions would probably 
    result in a very restrictive model.


\item \emph{Main contribution}: While I appreciate the interesting construction 
    based on the DPM of projected gammas (which itself could be of use for other 
    contexts beyond extremes?), one wonders about what would be the main added 
    value of it with respect to other existing Bayesian approaches 
    (\cite{boldi2007},\cite{guillotte2011},\cite{SaNa2014},\cite{hanson2017}).
    Keeping in mind the computational scope of Statistics and Computing, one 
    wonders if the main advantage is, for example, a computational one? This 
    needs to be clarified—especially in light of the previous comment on the 
    moment constraint. Indeed, one could use any of the previous approaches with 
    pseudo-angles based on the infinity norm.

We have clarified the distinction between our approach and the prior contributions
    mentioned.  Setting aside \cite{guillotte2011} which is a strictly bivariate
    model, the named approaches consider batch extrema and by construction must
    respect the moment constraint.  As noted, our approach is able to set that 
    aside, allowing consideration of a wider class of distributions of pseudo-angles.

    % \comment{Hanson is clearly different.  They're using Bernstein polynomials 
    % effectively defined on $\mathbb{S}_1^{d-1}$.  We can demonstrate that 
    % projected gammas built on $\mathbb{S}_{p}$, $p \geq 10$ and $\mathbb{S}_1$ 
    % create \emph{very} different gamma shape parameters, for data created by 
    % the \emph{same} distribution.  I need to check again the other references, 
    % but based on their follow-on comment, I'm guessing the previous point needs 
    % to be addressed before we can properly justify this point.}
    
    % \bruno{Based on our conversation, you need to look carefully into each one 
    % of these references. it seems that most of them do not develop a PoT 
    % approach. But also, we use a kernel whose natural support is the positive 
    % cone (without using transformations) to build a mixture WRT to a random
    % measure. This provides the flexibility of mixture for a methods that has 
    % computational advantages and scales to moderately large problems, as 
    % illustrated in the paper.}

    % \comment{\cite{Hanson2017}: Bernstein polynomials on $\mathbb{S}_1^{d-1}$, 
    % modelling multivariate Frechets.  \cite{boldi2007}: Marginal Frechets, also 
    % operating on $\mathbb{S}_1^{d-1}$.  \cite{guillotte2011}: A strictly bivariate model.  
    % \cite{SaNa2014}: }

\item \emph{Numerical accuracy}:  Keeping in mind the computational scope of 
    Statistics and Computing one would hope for more numerical work than what is 
    presented in §5.1. As far as I understand what is presented are single run 
    experiments for each $d$? Monte Carlo evidence on the performance of the 
    procedure would be appreciated. On top of this, the current numerical exercise
    ignores the uncertainty that stems from step 1 of the procedure 
    (i.e. estimating the margins).

To address this concern, we have revised the simulation study to feature the 
    average rise in energy score over baseline, averaged over 10 replications.  
    The simulated datasets are generated as mixtures of multivariate lognormals, 
    projected onto the unit sphere.  As you say, this paper focuses on estimating
    the spectral distribution, and ignores uncertainty from estimating marginal
    parameters.  We have clarified as such.
    

    % \comment{Regarding the uncertainty from step 1, I think we agreed that was 
    % basically intractible. We may need to explicitly specify as such, and that a 
    % practicioner must accept the numerical uncertainty that arises from step 1.  
    % That said, I seem to recall the numbers observed being relatively robust to 
    % choice of threshold (for large threshold), the difference mainly coming down 
    % to computation time.}
    
    % \bruno{I think we need to conduct a more comprehensive suite of experiments. 
    % Regarding the uncertainty coming from Step1, we just need to acknowledge that 
    % we are ignoring it.}


\item \emph{Prior and MCMC}: A lot is left to the reader in §5 in terms of
    prior specification and MCMC settings.  At least some further details would 
    be required.

    \comment{The form of the prior is specified.  I'm guessing he's referring to 
    the actual values used for computation?  $\bm{\mu}_0 = \bm{0}$, 
    $\nu = d + 50$, $\bm{\Psi} = \nu I_d$, and so on?}
    
    \bruno{You need to provide more details here.}

\item \emph{Future research}: §6 calls for a view of the authors on what 
    could be related exercises—or even extensions of this research and future 
    work. Using the proposed Bayesian multivariate PoT construction for 
    devising a regression model for an an extreme value response on an extreme 
    value covariate is one example of a natural follow-up; see 
    \cite{carvalho2022} for related ideas. Conditional versions of the angular 
    measure (e.g. \cite{carvalho2016}, \cite{castro2018}, \cite{escobar2018}, 
    \cite{mhalla2019}) would lead to a conditional multivariate 
    peaks-over-threshold model, where dependence between exceedances could 
    change along with a covariate. Also, perhaps the here proposed DPM of 
    projected gammas—as well as predictor-dependent versions of it—could be of 
    interest in other settings beyond angular measures?

    \comment{I'm guessing this guy is de Carvalho?}
    
    \bruno{I have no doubts that this referee is Carvalho. I'll spend some time 
    reading the papers that he mentions.}
    
\end{itemize}

\section*{Review 2}
\begin{itemize}
\item \emph{Lack of clarity}: The presentation of the probabilistic setting for the multivariate 
    PoT model (Section 2) and estimation method (Section 3) contains some inconsistencies and unclear
    points:
    \begin{enumerate}
    \item Please use a different notation for the random vector distributed according to $F$ 
        (denoted by $W$ in the paper) and for the Peaks-over-threshold limit, that is the standard
        Generalized Pareto vector, also denoted by $W$. This is not correct because the latter has 
        not the same distribution as the former. The confusion goes on throughout the paper, e.g. 
        the standardized variables in Section 3 (after a probability integral transform, Equation
        (3), are denoted by the letter $z$. Notice that the vector $W$ defined as a limit in Section 
        2 does not necessarily follow a standard Generalized Pareto distribution. The relation 
        (and the difference) between standard and non-standard is explained e.g. in [3], Proposition 4. 
        In order to have the representation $W = W_{\infty}V$ as in the paper, with 
        $W_{\infty} \sim \text{Pa}(1)$, one needs to assume that $W$ is indeed standard GP. In other 
        words it is correct to write $Z = RV$ where $Z$ is a properly standardized vector, not $W = RV$.

        \comment{Yeah.  W is a multivariate GPD vector; Z is a multivariate standardized Pareto vector.
        This needs to be fixed.}
        
        We thank the referee for pointing out the inconsistencies in the notation. We have corrected them 
        to make things clearer. We denote all random variables with capital letters and their values
        with small letters. The random variable corresponding to the conditional limiting distribution
        $H$ is now denoted as $Z$.

    \item About the standardization, Eq. (3): please introduce some background. In the paper it is
        introduced as \emph{’the standardization’} and it is unclear what is the relation between this 
        and the definition of the GPD (beginning of Section 2), unless one already knows about marginal
        standardization methods in multivariate EVT. In any case there are several ways to perform 
        marginal standardization in multivariate EVT.

        We have rewritten the material at the beginning of Section 3, explicitly mentioning the marginal
        univariate generalized Pareto, and how it is used to estimate the shape and scale parameters, 
        which are then used to standardize the observations.

    \item Section 3.2.1: Please provide some minimal background (or references) on the Dirichlet 
        Process Mixture, in particular about clusters of observations occurring naturally in this 
        model. This is indeed standard in Bayesian Non-parametrics, so there should be a precise 
        reference where it is explained.

        \bruno{Peter, you need to come up with additional, and more specific references here.}

        
\end{enumerate}

\item \emph{Organization of paper}:
    \begin{enumerate}
        \item End of Section 2, starting from \emph{A measure that is used to characterize 
            the strength of dependence...}, until the end: the discussion about the dependence 
            coefficient (Coles’s $\chi$) is quite disconnected from the rest at this point. It
            is not useful for Section 3. It is only used in the experiments (Section 5). If I 
            understand well, in Section 5, pairwise $\chi$’s are computed by Monte-Carlo simulation 
            in the whole model, which exemplifies the usefulness of the projected gamma model. 
            However these $\chi$’s may be computed in many different ways, the most simple being 
            the empirical version. It is unclear at this point what is the added value of the 
            model compared with an empirical method.

        \item The statistical model (projected Gamma family) is only introduced in the
            \emph{‘estimation’} section, whereas it is the main building block of the paper.

            We thank the referee for the suggestion to change the layout of the paper. We have
            re-written Section 3 to put a stronger emphasis on the presentation of the
            projected gamma kernel.
            
        \end{enumerate}

    \item \emph{Unclear Goals}: 
    \begin{enumerate}
        \item There is no proper justification in the paper for the focus on the infinity norm 
        and the associated sphere $\mathbb{S}_{\infty}^{d-1}$. In \cite{ferreira2014}, the focus
        is on the space of bounded processes, and this infinite dimensional sample space is 
        naturally associated with the infinity norm. However in the multivariate case here, the 
        setting is very different and there is no obvious reason why the max norm should be used. 
        Also, by changing the conditioning event, i.e. replacing $\lbrace W \not\leq b_{n}\rbrace$
        (notations of the paper, beginning of Section 2) with 
        $\lbrace \lVert W\rVert_p > t_n\rbrace$, one will obtain a product representation in the 
        limit of the form $\tilde{W} = RV_p$ where $R$ is Pareto and $V_p$ is located on the 
        $\mathcal{L}_p$-sphere. See e.g. one of Resnick’s books, where it is clearly stated that 
        the product form of the limit does not depend on the choice of the norm. I believe that 
        the projected Gamma family on the $L_p$ sphere with a DPM prior is anyways a nice 
        contribution, which is somewhat hidden by this additional projection on the infinity sphere.

        \bruno{We need to get hold of a copy of the right book to see what the result is. If the result 
        is so straightforward, then we need to decided what to do about it, but I am worried that
        it will require a moment constraint.}
        
        \item Another issue is that it is not clear in the present form of the paper, what is the
        added value of fitting this parametric model with the Bayesian framework and the Gibbs 
        sampler, if the output is mainly bivariate densities and extremal dependence coefficients 
        which could be obtained by standard non parametric methods with very low computational cost?

        \bruno{This is an unfair comment. We do use the full multivariate distributiont to come up
        with the results we present. I'll work on an answer to this one.}
    \end{enumerate}

\item \emph{Comparison with other methods}: The only comparison study displayed in the paper 
    takes place within the suggested model, only with different prior choice. This is not sufficient 
    for performance comparison in a methodological paper. One may e.g. think of comparing with the
    results obtained with the empirical angular measure or smoothed versions of it in terms of 
    extremal dependence and conditional survival probability.

    We have included for comparison some other models established on $\mathcal{S}_{1}^{d-1}$;
        notably the pairwise betas model developed in \cite{sabourin2013}, as well
        as the dirichlet distribution.  We have investigated several packages to 


    \bruno{Well, here we have to bite the bullet. Hopefully there some software that allows us
    to fit the empirical angular measure.}

    \item \emph{Experiments}:
        \begin{enumerate}
            \item Computational Evidence missing: The convergence of the Gibbs sampler is not 
            discussed in the paper.

            \bruno{This one was pointed out by the other referee. So, we need to come up with something.}

            \comment{\st{I'm not sure how to demonstrate convergence in the case of the DP.  I can
            I can demonstrate convergence for certain quantities (e.g., the hierarchical mean,
            and log-determinant of hierarchical covariance matrix, a given observation's distribution
            of parameters, etc.), but in the general stick-breaking configuration, there is no concern
            given to label switching.  The variational approach will probably find a label arrangement
            and stick with it.}}

            \comment{I'm working on adding log-density over time plots to demonstrate convergence.}
            
            \item Figure 2: The scores from the ‘baseline’ should outperform the scores of
            the fitted models, because the baseline consists of samples generated from the 
            true distribution. This is not the case in Figure 2, which deserves an explanation 
            (it might be a too small sample size, or some coding error at worst).

            \bruno{We need to think about this one.}

            \comment{I think it was an issue of understanding.  I've updated the code, remaking the plots
            is nearly done.}

        \end{enumerate}

    \item \emph{About the absence (or presence) of moments constraint}: The angular measure in
    multivariate extremes with respect to any norm has a moment constraint which precise form depends 
    on the chosen norm and on the standardization, see e.g. 
    \cite{beirlant2006} Chapter 8, Section 8.2.3, Equation 8.20. 
    This seems to contradict the author’s claim that there is no moments constraints in the models, 
    due to the choice of the $\mathcal{L}_{\infty}$-norm. Although it may be the case that their choice
    of standardization removes the moments constraint, I do not understand how the choice the 
    $\mathcal{L}_{\infty}$-norm helps here.

    Another way to see this: for $x \geq 1$, using the representation $W = W_{\infty}V$ as in the
    paper, with $W_{\infty}$ a standard Pareto variable, we get 
    \[\mathbb{P}\left[W_{\ell} > x\right] = \mathbb{E}\left[V_{\ell}\right] / x\]
    (see also Equation (2.20) in \cite{ferreira2014}, with $\omega_0 = 1$).  As a consequence if 
    the distribution of the marginal variable $W_{\ell}$ in the limit is imposed, then so is 
    $\mathbb{E}\left[V_{\ell}\right]$, namely 
    \[ \mathbb{E}\left[V_{\ell}\right] = \mathbb{P}\left[W_{\ell} > 1\right] \]
    It may be the case that, with the standardization Equation (3) in their paper, the marginal
    distributions, i.e. the distributions of the $W_{\ell}$’s in the limit are not entirely determined, 
    but this needs to be clarified.

    \comment{I seem to regard something in the intro to extremes book regarding the lack of a
    moment constraint on the $\mathcal{L}_{\infty}$ norm.  I'll look for it.}

    \bruno{This was dealt with already.}

    \item \emph{Energy Score, kernel function, flaw in the proof}: In the proof of proposition 3, 
    one needs to verify that $g$ is a negative definite kernel, that is, that for all $x_1,\ldots,x_n$
    and $a_1,\ldots,a_n$, \[ \sum_i\sum_ja_ia_jg(x_i,x_j)\leq 0,\]
    (with varying points $x_i,x_j$), not that \[\sum_i\sum_ja_ia_jg(x_1,x_2)\leq 0.\]  Thus the vector $e$
    which allows to break the sum into two parts in the current proof should in fact depend on $(i,j)$, 
    and the proof is not valid.

    \bruno{We need to rewrite the proof. I'll do that.}

    \comment{\st{Yeah, I think I messed this one up here.  I'm going to run simulations to verify it is
    a negative definite kernel, then try and figure out a better proof.  Too bad the trivial one
    didn't work.}}

    \comment{\st{I'm somewhat worried that they will regard the new proof as also problematic, as the second 
    term is dependent on the first term.  We could get around this by splitting $\sum_{i,j}a_ia_jg(x_i,x_j)$ 
    into sets belonging to particular face pairs (e.g., $x_i$ on face $a$, $x_j$ on face $b$) and sum over 
    permutations of $a,b$, but then there's no guarantee that a particular $a_i$ set within a particular 
    $a,b$ face set will sum to 0.}}

    \item \emph{Minor point 1}: About the word \emph{‘Projection’}, e.g. \emph{in 
    ‘To define an angular measure, we are interested in the direction of vectors in $\mathbb{R}_+^d$
    thus, we project them onto $[\ldots]$}: The angular variable variable is not really a projection 
    (at least not an orthogonal one) except when the $\mathcal{L}_2$ norm is used.

    \comment{Indeed, it's not an orthogonal projection.  Even the $\mathcal{L}_2$ norm doesn't 
    produce an orthogonal projection, unless using the trigonometric mapping.  I don't think that's
    a problem.}

    \bruno{I think we need to de-emphasize the word ``projection''. We will call it ``normalization''
    the first time we mention it, and then say that we will use the term ``projection''. It is a
    projection in the sense that it is the point in $\mathbb{S}_p^{d-1}$ that is closest to the original
    point, is it not?}

    \item \emph{Minor point 2}: The first sentence of Section 3 is vague: [\emph{‘Consider a 
    collection of observations $w_l, l = 1,\ldots,n$ that exhibit extreme behavior}].  Probably the
    authors mean that the underlying random vector W is regularly varying, or belongs to some domain 
    of attraction. However only specialists of EVT can guess that.

    \comment{we're using $l$ for iterating observations, rather than dimensions?  Yeah, it's probably
    bad wording.}

    \item \emph{Minor point 3}: (typos): Section 3, first paragraph: Some confusions between 
        $\ell \in \lbrace 1,\ldots, n\rbrace$ and $\ell \in \lbrace 1,\ldots,d\rbrace$

    \comment{same as previous.}
\end{itemize}

\bibliography{./refs}

\end{document}

% EOF 