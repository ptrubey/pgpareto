\section{Scoring criteria for distributions in ${\mathbb S}_\infty^{d-1}$\label{sec:evaluation}}

In order to assess and compare the estimation of a distribution on ${\mathbb S}_\infty^{d-1}$ we consider 
    the theory of proper scoring rules developed in \cite{gneiting2007}. As mentioned in 
    Section~\ref{subsec:projgamma}, our approach does not provide a density on 
    ${\mathbb S}_{\infty}^{d-1}$, restricting our ability to construct model selection criteria 
    to sample-based approaches.  To this end, we employ the \emph{energy score} criterion introduced
    therein.

The energy score criterion defined for a general probability distribution $P$, with finite expectation, 
    is developed as
    \begin{equation}
    \label{eq:es}
    \begin{aligned}
    &\hspace{-0.5cm}\text{S}_{\text{ES}}\left(P, \bm{x}_i\right) \\
    &=\text{E}_p \left[g\left(\bm{X}_i, \bm{x}_i\right)\right] -
                                    \frac{1}{2}\text{E}_p \left[g\left(\bm{X}_i,\bm{X}_i^{\prime}\right)\right],
    \end{aligned}
    \end{equation}
where $g$ is a kernel function. The score defined in Equation \eqref{eq:es} can be evaluated
  using samples from $P$, with the help of the law of large numbers.
  Moreover, Theorem~4 in \cite{gneiting2007}, states that if $g(\cdot,\cdot)$ is a negative 
  definite kernel, then $\text{S}(P,\bm{x})$ is a \emph{proper} scoring rule.  Recall that a 
  real valued function $g$ is a negative definite kernel if it is symmetric in its arguments, 
  and $\sum_{i=1}^n\sum_{j=1}^na_ia_jg(x_i,x_j)\leq 0$ for all positive integers $n$, and any 
  collection $a_1,\ldots,a_n\in{\mathbb R}$ such that  $\sum_{i = 1}^na_i = 0$.  

In a Euclidean space, these conditions are satisfied by the Euclidean distance \citep{berg1984}. 
    However, for observations on different faces of ${\mathbb S}_{\infty}^{d-1}$, the Euclidean 
    distance will under-estimate the geodesic distance, the actual distance required to travel 
    between the two points.  Let
\begin{equation*}
    {\mathbb C}_{\ell}^{d-1} = \lbrace \bm{x} : \bm{x} \in {\mathbb S}_{\infty}^{d-1}, x_{\ell} = 1\rbrace
\end{equation*}
    comprise the $\ell$th \emph{face}.  For points on the same face, the Euclidean distance 
    corresponds to the length of the shortest possible path in ${\mathbb S}_{\infty}^{d-1}$.  
    For points on different faces, the Euclidean distance is a lower bound for that length.

For a finite $p$, the shortest connecting path between two points in ${\mathbb S}_p^{d-1}$
  is the minimum geodesic; its length satisfying the definition of a distance.  Thus its
  length can be used as a negative definite kernel for the purpose of defining an energy
  score. Unfortunately as $p\to\infty$ the resulting surface ${\mathbb S}_{\infty}^{d-1}$
  is not differentiable, implying that routines to calculate geodesics are not readily
  available.  However, as ${\mathbb S}_{\infty}^{d-1}$ is a portion of a $d$-cube, we
  can borrow a result from geometry \citep{pappas1989} stating that the length of the
  shortest path between two points on a geometric figure corresponds to the length of a
  straight line drawn between the points on an appropriate unfolding, rotation, or 
  \emph{net} of the figure from a $d$-dimensional to a $d-1$-dimensional space.  The optimal 
  net will have the shortest straight line between the points, as long as that line is fully 
  contained within such net. As ${\mathbb S}_{\infty}^{d-1}$ has $d$ faces---each face 
  pairwise adjacent, there are $d!$ possible nets.  However, we are only interested in nets 
  that begin and end on the source and destination faces respectively, reducing the number 
  of nets under consideration to $\sum_{k = 0}^{d-2}\binom{d-2}{k}$.  This is still 
  computationally burdensome for a large number of dimensions.  
  However, we can efficiently establish an upper bound on the geodesic 
  length.  We use this upper bound on geodesic distance as the kernel function for
  the energy score.

To calculate the energy score we define the kernel 
    \[  
    g(\bm{a},\bm{b}) = \min_{\bm{c} \in{\mathbb C}_{\jmath}^{d-1}\cap{\mathbb C}_{\ell}^{d-1}}\left\{ 
        \pnorm{\bm{c}-\bm{a}}{2} + \pnorm{\bm{b}-\bm{c}}{2} \right\}.
    \]
    where $\bm{a} \in {\mathbb C}_{\ell}^{d-1}$, and 
    $\bm{b} \in {\mathbb   C}_{\jmath}^{d-1}$, for $\ell,\jmath\in \{1,\ldots, d\}$.
Evaluating $g$ as described requires the solution of a $(d-2)$-dimensional 
  optimization problem.  The following proposition provides a straightforward approach.
\begin{prop}\label{prop:rot}
    Let $\bm{a} \in {\mathbb C}_{\ell}^{d-1}$, and $\bm{b} \in {\mathbb C}_{\jmath}^{d-1}$, 
        for $\ell, \jmath \in \{1, \ldots , d\}$.  For $\ell\neq \jmath$, the 
  transformation $P_{\jmath\ell}(\cdot)$ required to rotate the $\jmath$th face along the 
  $\ell$th axis produces the vector $\bm{b}^\prime$, where
  \begin{equation}
    \label{eqn:rotation}
     \bm{b}^{\prime}_i = P_{\jmath\ell}(\bm{b}) = 
    \begin{cases}
        b_{i} &\text{for }i\neq \jmath,\ell\\
        1 &\text{for }i = \ell\\
        2 - b_{\ell} &\text{for }i = \jmath
    \end{cases}.
  \end{equation}
  Then $g(\bm{a},\bm{b}) = \pnorm{\bm{a} - \bm{b}^{\prime}}{2}$.
\end{prop}
{\em Proof:}
Notice that for 
  $\bm{c} \in {\mathbb C}_{\jmath}^{d-1}\cap{\mathbb C}_{\ell}^{d-1}$,
     $\pnorm{\bm{b} - \bm{c}}{2} =  \pnorm{\bm{b}^{\prime} - \bm{c}}{2}$. %and let
%\[\bm{e} = 
%    \argmin_{\bm{c} \in{\mathbb C}_{\jmath}^{d-1}\cap{\mathbb C}_{\ell}^{d-1}}\left\{ 
%        \pnorm{\bm{c}-\bm{a}}{2} + \pnorm{\bm{b}-\bm{c}}{2} \right\}.
%\]
We then have that
  \begin{equation*}
    \begin{aligned}
    g(\bm{a},\bm{b}) = &\min_{\bm{c} \in{\mathbb C}_{\jmath}^{d-1}\cap{\mathbb C}_{\ell}^{d-1}}\left\{ 
        \pnorm{\bm{c}-\bm{a}}{2} + \pnorm{\bm{b}-\bm{c}}{2} \right\} \\
     &\min_{\bm{c} \in{\mathbb C}_{\jmath}^{d-1}\cap{\mathbb C}_{\ell}^{d-1}}\left\{ 
        \pnorm{\bm{c}-\bm{a}}{2} + \pnorm{\bm{b}^{\prime}-\bm{c}}{2} \right\}\\
    &= \pnorm{\bm{a} - \bm{b}^{\prime}}{2}.
    \end{aligned}
  \end{equation*}
  The last equality is due to the fact that $\bm{a}$ and $\bm{b}'$ belong to the 
  same hyperplane. $\hfill\Box$
    
Using the rotation in Proposition \ref{prop:rot} we obtain the following result.
    \begin{prop}\label{prop:g}
    g is a negative definite kernel.
    \end{prop}
    {\em Proof:} For a given $n$ consider an arbitrary set of points
    $\bm{a}_1, \ldots , \bm{a}_n \in {\mathbb S}_\infty^{d-1}$, and 
    $\alpha_1, \ldots , \alpha_n \in {\mathbb R}$, such that $\sum_{i=1}^n
    \alpha _i = 0$. Then
    \[
    \sum_{i,\jmath} \alpha_i \alpha_\jmath g(\bm{a}_i, \bm{a}_\jmath) 
        = \sum_{i,\jmath} \alpha_i \alpha_\jmath 
    \|\bm{a}_i - \bm{a}_\jmath^\prime\|_2 \leq 0,
    \]
    where $\bm{a}_\jmath^\prime$ is defined as in Proposition \ref{prop:rot}. 
    The last equality holds as $\|\bm{x} - \bm{x}^\prime\|_2, \bm{x},
    \bm{x}^{\prime} \in {\mathbb R}^d$ is negative definite
    \citep{gneiting2007}$\hfill\Box$

Proposition \ref{prop:rot} provides a computational efficient way to evaluate the proper
    scoring rule $S_{\text{ES}}$ defined on ${\mathbb S}_\infty^{d-1}$, for each 
    observation. For the purpose of model assessment and comparison, we report the 
    average $S_{\text{ES}}$ taken across all observed data, and notice that the smaller 
    the score, the better.
    
% EOF
