
\section{Introduction}

The statistical analysis of extreme values focuses on inference for
rare events that correspond to the tails of probability distributions.
As such, it is a key ingredient in the risk assessment of
phenomena that can have strong societal impacts like floods, heat waves,
high concentration of pollutants, crashes in the financial markets,
among others. The fundamental challenge of extreme value theory (EVT) is
to use information, collected over limited periods of time, to
extrapolate to long time horizons. This sets EVT apart from most of
statistical inference, where the focus is on the bulk of the
distribution. Extrapolation to the tails of the distributions is
possible thanks to theoretical results that give asymptotic descriptions
of the probability distributions of extreme values. 

Inferential methods for the extreme values of univariate observations
are well established and software is widely available \cite[see, for example,][]{coles2001}. For variables 
in one dimension the application of EVT methods considers the asymptotic distribution of either the maxima 
calculated for regular blocks of data, or the values that exceed a certain threshold. The former leads to
a Generalized Extreme Value (GEV) distribution, that depends on three parameters. The latter leads to a 
Generalized Pareto (GP) distribution, that depends on a shape and a scale parameter. Likelihood-based 
approaches to inference can be readily implemented in both cases.
In the multivariate case the GEV theory is well developed \citep[see, for example][]{dehaan2006}, but the
inferential problem is more complicated, as there is no parametric representation of the GEV. This problem
is inherited by the peaks over threshold (PoT) approach and compounded by the fact that there is no unique definition of an 
exceedance of a multivariate threshold, as there is an obvious dependence on the norm that is used to
measure the size of a vector. 

During the last decade or so, much work has been done in the exploration of the definition 
  and properties of an appropriate generalization of the univariate GP distribution 
  to a multivariate setting.  To mention some of the papers on the topic, the work of
  \citep{rootzen2006} defines the generalized Pareto distribution, with further analysis on these classes
  of distributions presented in \cite{falk2008} and \cite{michel2008}.  A recent review of the state
  of the art in multivariate peaks over threshold modelling using generalized Pareto is provided in
  \cite{rootzen2018} while \cite{RoSeWa2018a} provides insight on the theoretical properties of possible
  parametrizations. These are use in \cite{KiRoSeWa2019} for likelihood-based models for PoT estimation.
  A frequently used method for describing the dependence
  in multivariate distributions is to use a copula. \cite{renard2007}, and \cite{falk2019} provide successful 
  examples of this approach in an EVT framework. \cite{ferreira2014} presents a constructive definition of the
  Pareto process, that generalizes the GP to an infinite dimensional setting. It consists of decomposing the p
  rocess into independent radial and angular components. Such an approach can be used in the finite dimensional 
  case, where the angular component 
  contains the information pertaining to the dependence structure of the random vector.
  Based on this definition, we present a novel approach for modelling  the angular component
  with families of distributions that provide flexibility and can be applied in a moderately large dimensional setting.
  Our focus on the angular measure is similar to that in \cite{SaNa2014} and \cite{HaCaCh2017}, that consider Bayesian 
  non-parametric approaches. Yet, our approach avoids the problem of dealing with the so called moment constraint by
  considering a constructive definition of the multivariate GP based on the infinity norm.
  
  The reminder of this paper is outlined as follows. Section~\ref{sec:multivariatepot} comprises a brief review of multivariate PoT, detailing the separation of the radial measure from the angular measure.
  Section~\ref{sec:methodology} details our approach for estimating the angular measure, based on projecting an arbitrary distribution supported in ${\mathbb R}_+^d$ onto unit hyper-spheres defined for $p$-norms. Section~\ref{sec:evaluation} develops criteria for model selection in the support of the angular measure.  Section~\ref{sec:results} explores the efficacy of the proposed approach on a set of simulated data, and, acknowledging the relevance of extreme value theory to climatological events~\citep{jentsch2007,vousdoukas2018,li2019}, estimates the extremal dependence structure for a measure of water vapor flow in the atmosphere, used for identifying atmospheric rivers.  Finally, Section~\ref{sec:conclusion} presents our conclusions and discussion.

Throughout the paper, we adopt the operators $\wedge$ to denote minima, and 
    the $\vee$ to denote maxima.  Thus $\wedge_i s_i = \min_i s_i$, and $\vee_i s_i = \max_i s_i$.  
    These operators can also be applied component-wise between vectors, such as
    $\bm{a}\wedge\bm{b} = (a_1\wedge b_1, a_2\wedge b_2,\ldots)$.  We use 
    uppercase to indicate random variables, lowercase to indicate points, and
    bold-face to indicate vectors or matrices thereof.
  
\section{A multivariate PoT model\label{sec:multivariatepot}}
%To develop a multivariate PoT model for extreme values consider a 
%$d$-dimensional random vector $\bm{W} = (W_1, \ldots ,W_d)$  with 
%marginal distributions $F_1(w_1), \dots , F_d(w_d)$, and define its
%standardization as $\bm{Z} = (Z_1, \ldots , Z_d)$, where $Z_i = 1/ (1 - F_i(W_i)), \; i=1, \ldots , d$.
%Following the discussion in \cite{goix2017} we assume a form of 
%regular variation, consisting on assuming the existence of a 
%limiting measure $\mu$ on $(0,\infty]^d$, such that.
%\begin{equation}\label{eq:reg-var}
%        \lim\limits_{n\to\infty}n\prob{\frac{1}{n}\bm{Z}\not\leq \bm{z}} = \mu\left([\bm{0},\bm{z}]^c\right),
%\end{equation}
%where the inequality and the interval are taken componentwise. 
%In this setting, $\mu$ is the asymptotic distribution of $\bm{Z}$ 
%in extreme regions, and it is  referred to as the \emph{exponent
%measure}. In addition, we assume that $\lim_{n\rightarrow\infty} 
%n\prob{\max\{Z_1, \ldots , Z_d\}>n} = 1$. Equation \ref{eq:reg-var}
%implies that $\mu$ has the homogeneity property $\mu(t\cdot) = 
%\frac{1}{t}\mu(\cdot)$i, and it can be shown that  $\mu$ 
%can be decomposed into a radial and an angular component. In fact, 
%for $\bm{z} = (z_1, \ldots , z_n) \in (0,\infty]^d$, let $r(\bm{z}) = 
%||\bm{z}||_\infty$ and $\bm{v}(\bm{z}) = \bm{z}/r(\bm{z})$. Define 
%$\Phi(B) = \mu(\{\bm{z} : r(\bm{z})>1, \bm{v}(\bm{z}) \in B\})$, for any
%$B\subset \mathbb{S}_{\infty}^{d-1}$, the positive orthant of the unit 
%sphere in infinite norm. $\Phi$ is referred to as the {\em angular 
%measure}. Then 
%\begin{equation}\label{eq:exp-measure}
%    \mu\left(\left\lbrace  \bm{z} \in [0,\infty)^d : \lVert \bm{z} \rVert_{\infty} > r,\;\bm{z}/\lVert\bm{z}\rVert_{\infty} \in B\right\rbrace \right) = r^{-1}\Phi(B).
%\end{equation}
%Thus, the radial and angular components of the exponent measure are
%independent. To obtain a multivariate peak over threshold model we 
%consider the radial and angular component of the random variable $\bm{Z}$
%denoted, respectively, as $R = \max_{i} Z_i$ and $\bm{V} = \bm{Z}/R$. 
%Then, using Equations \ref{eq:reg-var} and \ref{eq:exp-measure}, we have
%that

%To obtain a statistical approach to estimating $\mu$, we follow the approach proposed in .
%Define  , and assume that it is distributed as a standard Pareto, and thus $R>1$ with probability one. Also, define , and assume that $\bm{V}$ has probability distribution $\Phi$ concentrated in $\mathbb{S}_{\infty}^{d-1}$, the positive orthant of the unit sphere in infinite norm. Also, assume that $E(V_i)>0$, and that $ \bm{Z}$ and $R$ are independent. 
%We refer to $R$ and $\bm{V}$ as, respectively, the  of $\bm{Z}$. Under those assumptions,  the convergence stated in Equation \ref{eq:reg-var} holds with limit measure $\mu$ determined by

%\citep[e.g.][applied to the finite-dimensional case]{ferreira2014}
%\[
%    \mu\left(\left\lbrace  \bm{z} \in [0,\infty)^d : \lVert \bm{z} \rVert_{\infty} > r,\;\bm{z}/\lVert\bm{z}\rVert_{\infty} \in A\right\rbrace \right) = r^{-1}\Phi(A)
%\]
%for $r > 0$ and for Borel sets $A\subseteq \mathbb{S}_{\infty}^{d-1}$.

%    $\mu(tA) = \frac{1}{t}\mu(A)$.  Define  $R = \max_{i} Z_i$, and assume that $R>1$ with probability one. 
%    Also, define $V_i = Z_i/\max_i Z_i$, and assume that $E(V_i)>0$. These two assumptions together with the 
%    homogeneity of $\mu$ correspond to the second condition in Theorem 2.1 of
%    \cite{ferreira2014}, applied to the finite-dimensional case.
%    Thus, we have the factorization $\bm{Z} =  R\bm{V}$, where $R$ and $\bm{V}$  are independent. Moreover,
%    for some measure $\Phi$, $\mu(\cdot)$ is factorized as
%    \temp{\begin{equation}\label{mufact}
%        %\mu\left( \left\lbrace\bm{Z} : V \in A , R > r\right\rbrace \right) = r^{-1}\Phi(A).
%        \mu\left(\left\lbrace \bm{z}\in \mathbb{R}_+^d : R(\bm{z}) > r, V(\bm{z}) \in A\right\rbrace\right) = r^{-1}\Phi(A)
%    \end{equation}
%    for $R(\bm{Z}) > 0$, and Borel sets $A \subseteq \mathbb{S}_{\infty}^{d-1}$.  Hereafter $R$ is referred
%    to as the radial component, and $\bm{V}$ the angular component of $\bm{Z}$
%    \st{$R$ describes the radial component, and $\bm{V} \in {\mathbb S}_{\infty}^{d-1}$ the 
%    angular component of $\bm{Z}$.}}  Here ${\mathbb S}_\infty^{d-1}$ denotes the positive
%    orthant of the unit sphere in infinite norm.  $\Phi(\cdot)$ is referred to as the \emph{angular measure}.   
%    From Equation (\ref{mufact}) we see that $R$ is distributed as a standard Pareto random variable. Conditioning on $R$, we have that
%    \begin{equation} \label{MGP}
%        \prob{\bm{V} \in B \mid R > r} = \frac{r\prob{\bm{V} \in B, R > r}}{r\prob{R > r}}\hspace{0.2cm}
%            \xrightarrow[r\to\infty]{~} \hspace{0.2cm} \Phi(B),
%    \end{equation}
%as $\lim_{r\rightarrow\infty} r\prob{R>r} = 1$.
%    Thus, the angular measure corresponds to the distribution of $\bm{V}$, conditional on $R$ being large.
%    As such, it controls the dependence structure of $\bm{Z}$ in the tails. 
%    In fact, assuming that $E(V_i)>0$ 
%    and following \cite{ferreira2014}, in the limit the distribution function of $\bm{Z}$ is given as
%    \begin{equation*}
%        \prob{Z_1\leq z_1, \ldots ,Z_d\leq z_d} = \expect{\bigvee_{i=1}^d
%    \frac{V_i}{z_i\wedge 1}} - \expect{\bigvee_{i=1}^d
%    \frac{V_i}{z_i}}
%    \end{equation*}
%    where the expectation is taken with respect to $\Phi$
%In view of this, to obtain a PoT model we consider a limiting 
%distribution for $\bm{Z}$ that corresponds to the distribution
%of the product of two independent random variables $\tilde{R}$ 
%and $\tilde{\bm{V}}$, with distributions, respectively, standard 
%Pareto and $\Phi$. \cite{ferreira2014} refer to such model as a 
%simple  Pareto process indexed on a finite dimensional set.  We 
%seek a flexible model for the distribution
%of $\tilde{\bm{V}} \in {\mathbb S}_{\infty}^{d-1}$.

%An alternative approach to justify the above mentioned limiting
%distribution, that provides a link to the multivariate GEV and the
%theory of max-stable distributions, is discussed 
%in Remark 1 of \cite{RoSeWa2018a}. More specifically, 
%using the notation in Theorem 6 of \cite{RoSeWa2018a}, define $\bm{S} 
%=  \log(\tilde{\bm{V}})$.  Then, according to Definition 5, $\bm{S}$ 
%is a spectral  random vector. Taking  $E \sim \text{Exp}(1)$, 
%independent of  $\tilde{\bm{V}}$,  we have that $\bm{Y} = \bm{S} + E$  
%follows a multivariate GP, according  to Theorem 6.  Then, using 
%Proposition 4,  $\tilde{\bm{V}}\tilde{R} = e^{\bm{Y}}$ will also follow
%a multivariate GP. Notice that $\tilde{R} = e^E$ is distributed as 
%a standard Pareto random variable. This is one of several stochastic
%representations of a multivariate GP presented in \cite{RoSeWa2018a}.

To develop a multivariate PoT model for extreme values consider a 
$d$-dimensional random vector $\bm{W} = (W_1, \ldots ,W_d)$ with
cumulative distribution $F$. Following \cite{RoSeWa2018a}, assume 
that there exists sequences of vectors $\bm{a}_n$ and $\bm{b}_n$,
and a $d$-variate distribution $G$ such that 
$\lim_{n\rightarrow\infty} F^n(\bm{a}_n \bm{w} + \bm{b}_n) = 
G(\bm{w})$. $G$ is a $d$-variate generalized extreme value 
distribution. It follows that
\begin{equation*}
    \begin{aligned}
    \lim\limits_{n\rightarrow\infty} &\prob{\bm{a}^{-1}_n (\bm{W} - \bm{b}_n) 
   \leq \bm{w} \mid \bm{W} \not\leq \bm{b}_n)} \\ 
     &\hspace{0.5cm}= \frac{\log G(\bm{w} \wedge \bm{0}) 
    - \log G(\bm{w}) }{\log G(\bm{0})} = H(\bm{w}).
    \end{aligned}
\end{equation*}
$H$ a multivariate Pareto distribution, and corresponds to the joint
distribution conditional on exceeding a multivariate threshold. In addition to
a copula function, $H$ depends on two $d$-dimensional vectors of 
parameters that we denote as $\bm{\xi}$ for the shapes and $\bm{\sigma}$ 
for the scales. \cite{RoSeWa2018a} provide a number of stochastic 
representations for $H$.  In particular, denote as $Z$ a
random variable with distribution $H$ where $\bm{\xi}= 1$ and 
$\bm{\sigma} = 0$.  Then, Remark 1 justifies the representation in 
\cite{ferreira2014}, giving $\bm{Z} = \bm{V}R$
where $R$ and $\bm{V}$ are independent. $R = \|\bm{Z}\|_\infty$ is 
distributed as a standard Pareto random variable, and $\bm{V} = 
\bm{Z}/\|\bm{Z}\|_\infty$ is a random vector in  
$\mathbb{S}_{\infty}^{d-1}$, the positive orthant of the 
unit sphere under $\mathcal{L}_{\infty}$ norm, with distribution $\Phi$. 
This representation is central to the methods proposed in this paper.
$R$ and $\bm{V}$ are referred to, respectively, as the {\em radial} 
and {\em angular} components of $H$. The angular measure controls 
the dependence  structure of $\bm{Z}$  in the tails. In view of 
this, to obtain a  PoT model we seek a flexible model for the 
distribution of $\bm{V} \in {\mathbb S}_{\infty}^{d-1}$.

The approach considered in \cite{rootzen2018} focuses on the limiting
conditional distribution $H$. An alternative approach consists of assuming
that regular variation \cite[see, for example,][]{resnick2008extreme} holds for 
the limiting distribution of $\bm{W}$, implying
\[
    \lim\limits_{n\rightarrow\infty} n \prob{n^{-1} \bm{W}\in A} = 
    \mu(A),
\]
where $\mu$ is termed the exponent measure. $\mu$ has the homogeneity property
$\mu(tA) = t^{-1}\mu(A)$. Letting $\rho = \|W\|_p, p>0$ 
and $\bm{\theta} = \bm{W}/\rho$, define  $\Psi(B) = \mu(\{\bm{w} : \rho>1, 
\bm{\theta} \in B\})$, which is referred to as the angular measure. After 
some manipulations, we obtain that
\begin{equation}
\label{exponent-mu}
    \lim\limits_{r\rightarrow\infty} \text{Pr}\left[\bm{\theta}\in A \rvert \rho>r\right] = 
    \frac{\Psi(A)}{\Psi({\mathbb S}_p^{d-1})} .
\end{equation}
Thus, a model for the exponent measure induces a model for the limiting distribution
conditional on the observations being above a threshold defined with respect
to their $p$-norm. The constraint that all marginals of $\mu$ correspond to a
standard Pareto distribution leads to the so called moment constraints on $\Psi$. 
Inference for the limiting distribution of the exceedances
needs to account for the normalizing constant in Equation \ref{exponent-mu}, as well
the moment constraints. In addition, the moment constraints can impose unrealistic
symmetries in the data distribution.


%\subection{Pairwise Extremal Dependence\label{subsec:ped}}
%%A measure that is used to characterize the strength of the dependence,
%%in the tail, for two random variables $Z_1$ and $Z_2$, with marginal
%%distributions $F_1$ and $F_2$ is given by \citep{coles2001}.
%%\[
%%    \chi_{12} = \lim\limits_{u\uparrow 1} \prob{F_1(Z_1)>u\mid F_2(Z_2)>u}  
%%\]
%%$\chi_{12}$ provides information about the distribution of extremes for the variable $Z_1$
%%given that $Z_2$ is very large.  When $\chi_{12}>0$, $Z_1$ and $Z_2$ 
%%are said to
%%be asymptotically dependent, otherwise they are asymptotically
%%independent. The following result provides the asymptotic dependence coefficient between two components of $\bm{Z}$
%%after PoT limit. 
%%\begin{prop}\label{ppchi}
%%Suppose that $\bm{Z} = R\bm{V}$ with $R\sim Pa(1)$,
%%$\prob{V_\ell > 0} = 1$ and $\expect{V_\ell}$ exists, for $\ell=1, \ldots ,d$, then
%%\begin{equation}
%%    \label{eqn:chi_ij}
%%	\chi_{\jmath\ell} = \expect{\frac{V_\jmath}{\expect{V_\jmath}} \wedge \frac{V_\ell}{\expect{V_\ell}}}
%%\end{equation}
%%\end{prop}
%%{\em Proof:}
%%Denote as $F_\ell$ the marginal distribution of $Z_\ell$. To obtain $\chi_{\jmath\ell}$ we need $Pr(Z_\jmath>z_\jmath,Z_\ell>z_\ell)$, where $z_\ell =
%%F_\ell^{-1}(u) = \expect{V_\ell}/(1 - u), \;\ell=1,\dots, d$.
%%Using the fact that
%%$V_\ell>0, \forall \ell$ almost surely, we have that the former is equal to
%%\begin{equation*}
%%    \begin{aligned}
%%    &\hspace{-0.5cm}\prob{R>\frac{z_\jmath}{V_\jmath}\vee\frac{z_\ell}{V_\ell}}
%%    = \expect{1\wedge\left(\frac{z_\jmath}{V_\jmath}\vee\frac{z_\ell}{V_\ell}\right)^{-1}}\\
%%    &= \expect{\frac{V_\jmath}{z_\jmath}\wedge\frac{V_\ell}{z_\ell}}
%%    = (1 - u)\expect{\frac{V_\jmath}{\expect{V_\jmath}}\wedge\frac{V_\ell}{\expect{V_\ell}}},
%%    \end{aligned}
%%\end{equation*}
%%where the second identity is justified by the fact that $V_i$ is
%%bounded and $z_i\rightarrow\infty$. The proof is completed by noting
%%that $\prob{F_i(Z_i)>u} = 1 - u. \hfill\Box$
%%
%%%\subsection{Multivariate Conditional Survival Function\label{subsec:condsurv}}
%%Equation \ref{eqn:chi_ij} implies that $\chi_{\jmath\ell}>0$, and so, no asymptotic independence is possible under our proposed model. For the analysis of extreme values it is of interest to calculate the multivariate conditional survival function. The following result provides the relevant expression, as a function of the angular measure.
%%\begin{prop}
%%Assume the same conditions of Proposition \ref{ppchi}. 
%%Let $\alpha \subset \{1, \ldots ,d\}$ be a collections of indexes. 
%%Then     
%%\begin{equation} \label{eqn:condsurv}
%%    \begin{aligned}
%%    &\hspace{-0.5cm}\mathrm{Pr}\left[\bigcap_{\ell \in \alpha} Z_\ell > z_\ell \mid \bigcap_{\ell\not\in\alpha} Z_\ell > z_\ell\right] \\
%%    &= \frac{\expect{\bigwedge_{k = 1}^d 1\wedge \frac{V_k}{z_k}}}{\expect{\bigwedge_{k \not\in\alpha}1\wedge\frac{V_k}{z_k}}}.
%%    \end{aligned}
%%  \end{equation}
%%\end{prop}  
%%The proof uses a similar approach to the proof of Proposition \ref{ppchi}.
%%
  % EOF
