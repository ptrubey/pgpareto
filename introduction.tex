
\section{Introduction}
The statistical analysis of extreme values focuses on inference for
rare events that correspond to the tails of probability distributions.
As such, it is a key ingredient in the assessment of the risk of
phenomena that can have strong societal impacts like floods, heat waves,
high concentration of pollutants, crashes in the financial markets,
among others. The fundamental challenge of extreme value theory (EVT) is
to use information, collected over limited periods of time, to
extrapolate to long time horizons. This sets EVT apart from most of
statistical inference, where the focus is on the bulk of the
distribution. Extrapolation to the tails of the distributions is
possible thanks to theoretical results that give asymptotic descriptions
of the probability distributions of extreme values. 

Inferential methods for the extreme values of univariate observations
are well established and software is widely available \cite[see, for
example,][]{coles2001}. For variables in one dimension the application of EVT methods consider the asymptotic distribution of either the maxima calculated for regular blocks of data, or the values that exceed a certain threshold. The former leads to a Generalized Extreme Value (GEV) distribution, that depends on three parameters. The latter leads to a Generalized Pareto (GP) distribution, that depends on a shape and a scale parameters. Likelihood-based approaches to inference can be readily implemented in both cases.
In the multivariate case the GEV theory is well developed \citep[see, for example][]{dehaan2006}, but the inferential problem is more complicated, as there is no parametric representation of the GEV. This problem is inherited by the PoT approach and compounded by the fact that there is no unique definition of an exceedance of a multivariate threshold, as there is an obvious dependence on the norm that is used to measure the size of a vector. 

During the last decade or so much work has been done in the exploration of the definition and properties of an
  appropriate generalization of the univariate GP distribution 
  to a multivariate setting.  To mention some of the papers on the topic, the work of
  \cite{rootzen2006} defines the generalized Pareto distribution, with further analysis on these classes
  of distributions presented in \cite{falk2008} and \cite{michel2008}.  A recent review of the state
  of the art in multivariate peaks over threshold modelling using generalized Pareto is provided in
  \cite{rootzen2018} while \cite{RoSeWa2018a} provides insight on the theoretical properties of possible parametrizations. 
  A frequently used method for describing the dependence
  in multivariate distributions is to use a copula. \cite{renard2007,deng2011,falk2019}, provide successful examples of this approach in an EVT framework. {\bf is \cite{deng2011} really a relevant reference? It looks weird. Can we say something more about the copula approach? I confess my ignorance here.}
  \cite{ferreira2014} presents a constructive definition of the Pareto process, that generalizes the GP to an infinite dimensional setting. It consists of decomposing the process into independent radial and angular components. Such approach can be used in the finite dimensional case, where the angular component 
  contains the information pertaining to the dependence structure of the random vector.
  Based on this definition, we present a novel approach for modelling  the angular component
  with families of distributions that provide flexibility and can be applied in a moderately large dimensional setting.
  Our focus on the  angular measure is similar to that in \cite{SaNa2014} and \cite{HaCaCh2017}, that consider Bayesian non-parametric approaches. Yet, our approach avoids the problem of dealing with the so called moment constraint by considering a constructive definition of the multivariate GP based on the infinity norm.
  
  The reminder of this paper is outlined as follows. {\bf We need a roadmap. Part of the roadmap should mention the analysis of the IVT. Basically condensing the paragraph below in a couple of sentences.}

To motivate the topic and our proposed solution, we acknowledge the relevance of extreme analysis to
  climatological events.  Outcomes such as heat waves, extreme precipitation, storm surges, and
  such present a similar challenge to investigators, in that the events of interest can be very destructive,
  and occur in probability at the tail end of \emph{normal} behavior.  In this fashion, a model built
  to capture \emph{normal} behavior may struggle to adequately represent extreme behavior.  More
  specialized tools become necessary; especially so in a time of intensifying extreme weather
  events~\citep{jentsch2007,vousdoukas2018,li2019}.  To this end, we use the integrated water vapor
  transport (\emph{IVT}) data for observing atmospheric rivers as an example application for our proposed method.
  
Throughout the paper, we adopt the operators $\wedge$ to denote minima, and the $\vee$
  to denote maxima.  Thus $\wedge_i x_i = \min_i x_i$, $\vee_i x_i = \max_i x_i$.  These operators can
  also be applied component-wise between vectors, such as $\bm{a}\wedge\bm{b} = (a_1\wedge b_1, a_2\wedge b_2,\ldots)$.
  
\section{A multivariate PoT model}
%\subsubsection{Univariate Maxima}
%Regarding the asymptotic behavior of extreme events, there are a couple major strategies for conducting
%  inference---developing probabilistic estimates of extreme behavior.  First developed is the theory
%  of a limiting distribution on maxima---the largest observation from a sample.  For a sample $\bm{x}$
%  where $\bm{x} = (x_1,\ldots,x_n)$ represents a sequence of $n$ independent random variables from a
%  distribution function $F$, the distribution of the maximum $M_n$ of this sequence can be derived
%  as:
%  \begin{equation*}
%    \begin{aligned}
%      \text{Pr}\left[M_n\leq z\right] &= \text{Pr}\left[X_1 \leq z, \ldots, X_n \leq z\right]\\
%        &= \text{Pr}\left[X_1\leq z\right]\times\ldots\times\text{Pr}\left[X_n\leq z\right]\\
%        &= F(z)^n.
%    \end{aligned}
%  \end{equation*}
%  In situations where $F$ is unknown, we can seek to approximate the behavior of $F^n$ as
%  $n\rightarrow\infty$.  To ensure this does not degenerate to a point mass, we need to consider a
%  standardized sequence of maxima. If this sequence stabilizes as $n$ increases, the a limiting
%  distribution exists.  More specifically, if there exists some sequence of constants $a_n > 0$, $b_n$
%  such that:
%  \begin{equation*}
%    \text{Pr}\left[\frac{M_n - b_n}{a_n} \leq z\right] \stackrel{d}{\rightarrow} G(z)
%  \end{equation*}
%  as $n\rightarrow\infty$, then $G$ is a max-stable distribution, and $F$ is in the domain of
%  attraction of that max stable distribution.  \cite{frechet1927} originates the
%  field, identifying two limiting forms of max-stable distributions, which would become known as the
%  Fr{\'e}chet and Weibull distributions. \cite{weibull1951} expands the analysis of the Weibull
%  distribution.  \cite{fisher1928}, in what is now called the \emph{Fisher-Tippett Theorem}, identifies
%  the Fr{\'e}chet and Weibull distributions, along with an as then unnamed third form, as the three
%  limiting forms of the distribution of the maxima of a sample.  \cite{gumbel1935,gumbel1942} offer
%  analysis of that third form, now known as the Gumbel distribution.  Later works, including
%  \cite{jenkinson1955} reparameterize all three forms as special cases of a single unifying form,
%  the generalized extreme value distribution, \emph{GEV}:
%  \begin{equation*}
%    \label{eqn:gev}
%    F(m \mid \mu, \sigma, \xi) = \exp\left\lbrace-\left[1 + \xi\left(\frac{x - \mu}{\sigma}\right)\right]_{}^{-1/{\xi}}\right\rbrace.
%  \end{equation*}
%  Thus distributions in the domain of attraction of an EVD (Gumbel, Fr{\'e}chet, or Weibull) will be
%  in the domain of attraction of the GEV.

%As this distribution specifies asymptotic behavior for the maximum of a set of observations,
%  inference assuming this distribution requires we that specify some block of data, and for the block
%  report only the maximum.  A series of these blocks yielding a series of maxima allows us to conduct
%  inference about the parameters of the distribution.  Taking only the
%  maximum in a block of observations necessarily leads to the reduction of our sample size by a factor
%  of $1/\text{block size}$. In problems where the data occur in natural blocks, such as an hourly time
%  series where a natural block might be a day, this might be appropriate.  There is an implicit
%  violation of the assumption of independence within a block, in most cases, but that violation is
%  generally ignored. In data without a natural block, it may be difficult to justify an induced
%  artitifial block. Moreover, by reducing available information, retaining only one datum for each
%  block increases the variability of the parameter estimates.

%\subsubsection{Univariate exceedences over a high threshold}
%The \emph{Pickands-Balkema-de Haan Theorem}~\citep{balkema1974,pickands1975} offers us another means
%  of conducting inference that will be less wasteful of data.  It states that if the original distribution
%  $F$ is in the domain of attraction of the GEV, then excesses over a high threshold $u$ can be well
%  modelled using the Generalized Pareto distribution.  Again, let $X$ follow some distribution function
%  $F$. It follows that:
%  \begin{equation*}
%    \text{Pr}\left[X > u + y\mid X > u\right] = \frac{1 - F(u + y)}{1 - F(u)}
%  \end{equation*}
%  for $y > 0$.  If $F$ is in the domain of attraction of an EVD, then
%  \begin{equation*}
%    \lim\limits_{u\to u^{\prime}}\text{Pr}\left[X > u + y\mid X > u\right] = H(y)
%  \end{equation*}
%  where $u^{\prime}$ represents some upper boundary of $X$ has a functional form---the survival
%  function of a Pareto distribution. We approximate this asymptotic relationship by setting some
%  large threshold $u$, and then threshold exceedences $Y = X - u$ are modelled under the distrribution
%  function
%  \begin{equation*}
%    \label{eqn:gp}
%    H(y) = 1 - \left(1 + \xi\frac{y}{\sigma}\right)^{-\frac{1}{\xi}}.
%  \end{equation*}
%  This defines the generalized Pareto family of distributions.  Thus, if block maxima have a
%  limiting distribution $G$ within the EVD family, then threshold exceedances for a sufficiently
%  high threshold have a limiting distribution $H$ within the Generalized Pareto (GP) family.  We can
%  identify $\sigma$ as a scale parameter, but $\xi$ deserves special mention as the extremal index.
%  We can interpret the Pareto tail probability as
%  \begin{equation*}
%    \lim\limits_{t\to\infty}\frac{1 - F(ty)}{1 - F(t)} = y^{-\frac{1}{\xi}}
%  \end{equation*}
%  for $y > 1$.  Importantly, $\xi$ from the limiting distribution of excesses over a threshold is the
%  same parameter, and has the same value as the $\xi$ from the limiting distribution of the maximum
%  observation of a sample \citep{dehaan2006}.

%One other point of note is that for time series, the data are temporally correlated.  Thus the implicit
%  assumption of independence for those observations in excess of a threshold is violated.  One means
%  of ameliorating the effect of this correlation is to decluster the data---and one method of declustering
%  is, for a series of observations in excess of the threshold, to keep only the maximum observed value
%  in the series.

%\subsubsection{Multivariate threshold EVT}

%The generalization of EVT to multiple dimensions presents a unique challenge in the case of maxima---how
%  do we define the maximum of a sample?  How can we order observations in a multivariate scenario?
%  \cite{rootzen2006} offers a brief review of the subject.  One possible treatment sidesteps this
%  conundrum by using component maxima---let $\bm{M}_n = \left(\vee_i X_{i1},\ldots,\vee_i X_{id}\right)$.
%  Then the multivariate GEV establishes the limiting behavior of $\bm{M}_n$ as $n\to\infty$.  Note
%  that $\bm{M}_n$ need not be in the observed $\left\lbrace\bm{X}_i; i = 1,\ldots,n\right\rbrace$.
%  Then, if there is a set of constants $\bm{a}_n > \bm{0}$, $\bm{b}_n$ such that
%  \begin{equation*}
%    \lim\limits_{n\to\infty}\text{P}\left(\bigcap_{\ell = 1}^d
%      \frac{M_{n\ell} - b_{n\ell}}{a_{n\ell}} \leq x_{\ell}\right) = G(\bm{x})
%  \end{equation*}
%  with non-degenerate marginals, then $G(\bm{x})$ is a \emph{multivariate extreme value distribution}.
%  The marginal distribution of $M_{nl}$ is by construction a GEV with the parameters $a_{n\ell}, b_{n\ell}, \xi$.
%  With the existence of the marginal distributions established as a consequence of the existence of the
%  GEV, multivariate EVT then splits into two components---estimation of the marginal parameters, and
%  estimation of the dependence structure.  This separation is born out extensively in the literature.

To develop a multivariate PoT model for extreme values consider a $d$-dimensional random vector $\bm{Z} \in {\mathbb R}^d_+ = [0,+\infty)^d$, the positive $d$-dimensional cone. Following the discussion in \cite{goix2017} we assume a form of regular variation, consisting on assuming the existence of a limiting measure $\mu$, such that.
\[
    \lim\limits_{n\to\infty}n\text{Pr}\left[\frac{1}{n}\bm{Z}\geq \bm{z}\right] = \mu\left([\bm{0},\bm{z}]^c\right),
\]
where the inequality and the interval are taken componentwise. 
In this setting, $\mu$ is the asymptotic distribution of $\bm{Z}$ in extreme regions, and it is referred to as the
  \emph{exponent measure}. Under this condition $\mu$ features the homogeneity property $\mu(tA) = \frac{1}{t}\mu(A)$. 
Define that $R = \max_{i} Z_i$, and assume that $R>1$ with probability one. Also, define $V_i = Z_i/\max_i Z_i$, and assume that $E(V_i)>0$.
These two assumption together with the homogeneity of $\mu$ correspond to the second condition in Theorem 2.1 of
\cite{ferreira2014}, applied to the finite-dimensional case.
Thus, we have the factorization $\bm{Z} =  R\bm{V}$, where $R$ and $\bm{V}$  are independent. Moreover,
for some measure $\Phi$, $\mu(\cdot)$ is factorized as
\begin{equation}\label{mufact}
    \mu\left( \left\lbrace\bm{Z} : V \in A , R > r\right\rbrace \right) = r^{-1}\Phi(A).
\end{equation}
$R$ describes the radial component, and $\bm{V} \in \mathcal{S}_{\infty}^{d-1}$ the 
angular component of $\bm{Z}$.  Here ${\mathbb S}_\infty^{d-1}$ denotes the positive
orthant of the unit sphere in infinite norm.  $\Phi(\cdot)$ is referred to as the \emph{angular measure}.   
From Equation (\ref{mufact}) we see that $R$ is distributed as a standard Pareto random variable. Conditioning on $R$, we have that
\begin{equation} \label{MGP}
    \text{Pr}\left[\bm{V} \in A \mid R > r\right]
      = \frac{r\text{Pr}\left[\bm{V} \in A, R > r\right]}{r\text{Pr}[R > r]}\hspace{0.2cm}
      \xrightarrow[r\to\infty]{~} \hspace{0.2cm} \frac{\Phi(A)}{\Phi(\mathcal{S}_{\infty}^{d-1})}.
\end{equation}
 Thus, the angular measure corresponds to the distribution of $\bm{V}$, conditional on $R$ being large.
 As such, it controls the dependence structure of $\bm{Z}$ in the tails. In fact, following \cite{ferreira2014} the distribution function of $\bm{Z}$ is given as
\begin{equation*}
	\text{Pr}[Z_1\leq z_1, \ldots ,Z_d\leq z_d] = E\left(\bigvee_{i=1}^d
\frac{V_i}{z_i\wedge 1}\right) - E\left(\bigvee_{i=1}^d
\frac{V_i}{z_i}\right) 
\end{equation*}
where the expectation is taken with respect to $\Phi$.
In view of this, to obtain a PoT model for $\bm{Z}$, we seek 
a flexible distribution on $\bm{V} \in \mathcal{S}_{\infty}^{d-1}$.

\cite{ferreira2014} refer to the model defined in Equation \eqref{MGP} as a simple 
Pareto process indexed on a finite dimensional set. An alternative approach to 
define such distribution, that provides a link to the multivariate GEV, is discussed 
in Remark 1 of \cite{RoSeWa2018a}. More specifically, 
using the notation in Theorem 6 of \cite{RoSeWa2018a}, define $\bm{S} = \log(\bm{V})$. 
Then, according to Definition 5, $\bm{S}$ is a spectral random vector. Taking 
$E \sim \text{Exp}(1)$, independent of $\bm{V}$, we have that $\bm{Y} = \bm{S} + E$ 
follows a multivariate GP, according to Theorem 6.  Then, using Proposition 4, 
$\bm{Z} = e^{Y}$ is also a multivariate GP. Notice that $R = e^E$ is distributed as 
a standard Pareto random variable.

A measure that is used to characterize the strength of the dependence,
in the tail, for two random variables $W_1$ and $W_2$, with marginal
distributions $F_1$ and $F_2$ is given by
\[	\chi_{12} = \lim_{u\uparrow 1} \text{Pr}[ F_1(W_1)>u|F_2(W_2)>u]  \]
\citep{coles2001}.
$\chi_{12}$ provides information about the distribution of extremes for the variable $W_1$
given that $W_2$ is very large.  When $\chi_{12}>0$, $W_1$ and $W_2$ 
are said to
be asymptotically dependent, otherwise they are asymptotically
independent. The following result provides the asymptotic dependence coefficient for two components of $\bm{Z}$. 
\begin{prop} \label{ppchi}
Suppose that $\bm{Z} = R\bm{V}$ with $R\sim Pa(1)$,
$\text{Pr}[V_i>0] = 1$ and $E(V_i)$ exists ,
$i=1, \ldots ,d$, then
\begin{equation}
	\chi_{ij} = E\left(\frac{V_i}{E(V_i)} \wedge 
\frac{V_j}{E(V_j)}\right).
\end{equation}
\end{prop}
{\em Proof:}
Denote as $F_i$ the marginal distribution of $Z_i$. To obtain $\chi_{ij}$ we need $Pr(Z_1>z_1,Z_2>z_2)$, where $z_i =
F_i^{-1}(u) = E(V_i)/(1 - u), \;i=1,\dots, d$.
Using the fact that
$V_i>0, \forall i$ almost surely, we have that the former is equal to
\[	\text{Pr}\left[R>\frac{z_i}{V_i}\vee\frac{z_2}{V_j}\right] =
E\left(1\wedge\left(\frac{z_i}{V_i}\vee\frac{z_j}{V_j}\right)
^{-1} \right)
= E\left( \frac{V_i}{z_i}\wedge\frac{V_j}{z_j}
\right) = (1 - u)E\left(
\frac{V_i}{EV_i}\wedge\frac{V_j}{EV_j}
\right) ,
\]
where the second identity is justified by the fact that $V_i$ is
bounded and $z_i\rightarrow\infty$. The proof is completed by noting
that $Pr(F_i(Z_i)>u ) = 1 - u.\hfill\Box$

For the analysis of extreme values it is of interest to calculate the multivariate conditional survival function. The following result provides the relevant expression, as a function of the angular measure.
\begin{prop}
Assume the same conditions of Proposition \ref{ppchi}. 
Let $\alpha \subset \{1, \ldots ,p\}$ be a collections of indexes. 
Then     
\begin{equation} \label{eqn:condsurv2df}
    \text{Pr}\left[\bigcap_{l \in \alpha} Z_l > z_l \right|\left. \bigcap_{l\not\in\alpha} Z_l > z_l\right] =
      \frac{E\left(\bigwedge_{k = 1}^d 1\wedge \frac{V_k}{z_k}\right)}{E\left(\bigwedge_{k \not\in\alpha}1\wedge\frac{V_k}{z_k}\right)}.
  \end{equation}
\end{prop}  
The proof uses a similar approach to the proof of Proposition \ref{ppchi}.


%\subsection{Integrated Vapor Transport}
%Atmospheric rivers are temporary events, where large elongated regions of high concentrations of
%  water vapor are developed in the atmosphere and carry huge amounts of water potentially thousands
%  of miles.  The amount of water in transit during these events can dwarf that of terrestrial rivers.
%  For the targeted region, the atmospheric river can represent a significant portion of the
%  precipitation the region will experience.  Thus, such events are of great interest to meteorologists,
%  hydrologists, actuaries, land-use planners, and even farmers.  The ability to estimate the likely magnitude
%  and distribution of such events is thus of great importance~\citep{ralph2013,ralph2018}.

%One metric by which we might identify and declare atmospheric rivers is the integrated water vapor
%  transport, or \emph{IVT}.  This value represents the total amount of water
%  vapor being transported in an atmospheric column---that is, a column of the troposphere above a given
%  area of the surface of the Earth. Raw data are measured by dropsondes---measurement devices dropped
%  out of aircraft, taking measurements as they fall to the earth.~\citep{ralph2017}  A data vector
%  includes an estimate of IVT at grid cell at a period in time; where a grid cell specifies the surface
%  of the earth under the atmospheric column.  We have two such datasets, in differing spatial
%  resolutions---the lower resolution data splits the coast of california into 8 grid cells, while the
%  higher resolution does so into 47 grid cells~\citep{guan2015}.

%As IVT tracks water vapor in the atmosphere, extreme values in IVT can lead to
%  extreme precipitation events.  Atmospheric rivers are necessary events, in that they carry the
%  bulk of precipitation that California recieves, but extreme precipitation events can be extremely
%  destructive as well.  Thus, we are interested in the distribution of extreme precipitation events,
%  including their depenedence structure.  This tells us about the probability of jointly extreme
%  events---where IVT is extreme in two or more grid cells, as well as identifying particular
%  configurations of IVT that may be anomalous.  In California, groundwater is a managed resource.
%  Extreme precipitation events significantly affect groundwater, so anomalous precipitation events
%  can produce anomalous configurations of groundwater, leading to destructive flooding.

%We are specifically interested in extremal dependence---the relationship between the upper tails of
%  dimensions of the distribution.  In this case, that means the relationship between extreme values
%  in different grid cells.  We are going to be looking at point-in-time behavior rather than
%  considering the time series nature of the data---that relationship may come later.

%As we are interested in the extremal dependence, it makes sense that we would choose to represent
%  this data using tools from extreme value theory.  Extreme value theory, or \emph{EVT}, seeks to
%  model and assess probability of observing extreme events.  Such a topic is applicable generally,
%  but it finds particularly strong use among such fields as finance \citep{allen2013},
%  climatology \citep{trepanier2018}, and insurance \citep{beirlant1994}.  In these fields, extreme
%  events may represent significant loss to the body commissioning the study.  For instance, an
%  insurance company might commission a study on extreme weather events, as an extreme weather event
%  localized to a particular region could cause a spike in claims from that region.  Extreme value
%  theory offers us a tool set for making inference about the tails of a distribution, without having
%  observed said tails.  For
%  instance, with an extreme weather event like flooding, we can make predictions about return
%  levels---the average time until an observation of a particular magnitude occurs---without having seen
%  an observation of that magnitude, or having observed that long.  In the context of our motivating
%  example, extreme values in the IVT may represent the formation of an atmospheric river, which has
%  dramatic effects on precipitation and ground water.  The formation of an IVT may lead to flooding
%  and other negative effects, but is also contributes necessary water for irrigation and agriculture.
%  Development of statistical tools regarding these extreme events becomes necessary for making informed
%  decisions.






  % EOF
