
\section{Introduction}

The statistical analysis of extreme values focuses on inference for
rare events that correspond to the tails of probability distributions.
As such, it is a key ingredient in the risk assessment of
phenomena that can have strong societal impacts like floods, heat waves,
high concentration of pollutants, crashes in the financial markets,
among others. The fundamental challenge of extreme value theory (EVT) is
to use information, collected over limited periods of time, to
extrapolate to long time horizons. This sets EVT apart from most of
statistical inference, where the focus is on the bulk of the
distribution. Extrapolation to the tails of the distributions is
possible thanks to theoretical results that give asymptotic descriptions
of the probability distributions of extreme values. 

Inferential methods for the extreme values of univariate observations
are well established and software is widely available \cite[see, for example,][]{coles2001}. For variables in one dimension the application of EVT methods considers the asymptotic distribution of either the maxima calculated for regular blocks of data, or the values that exceed a certain threshold. The former leads to a Generalized Extreme Value (GEV) distribution, that depends on three parameters. The latter leads to a Generalized Pareto (GP) distribution, that depends on a shape and a scale parameters. Likelihood-based approaches to inference can be readily implemented in both cases.
In the multivariate case the GEV theory is well developed \citep[see, for example][]{dehaan2006}, but the inferential problem is more complicated, as there is no parametric representation of the GEV. This problem is inherited by the PoT approach and compounded by the fact that there is no unique definition of an exceedance of a multivariate threshold, as there is an obvious dependence on the norm that is used to measure the size of a vector. 

During the last decade or so much work has been done in the exploration of the definition and properties of an
  appropriate generalization of the univariate GP distribution 
  to a multivariate setting.  To mention some of the papers on the topic, the work of
  \cite{rootzen2006} defines the generalized Pareto distribution, with further analysis on these classes
  of distributions presented in \cite{falk2008} and \cite{michel2008}.  A recent review of the state
  of the art in multivariate peaks over threshold modelling using generalized Pareto is provided in
  \cite{rootzen2018} while \cite{RoSeWa2018a} provides insight on the theoretical properties of possible parametrizations. 
  A frequently used method for describing the dependence
  in multivariate distributions is to use a copula. \cite{renard2007,falk2019}, provide successful examples of this approach in an EVT framework. \cite{ferreira2014} presents a constructive definition of the Pareto process, that generalizes the GP to an infinite dimensional setting. It consists of decomposing the process into independent radial and angular components. Such approach can be used in the finite dimensional case, where the angular component 
  contains the information pertaining to the dependence structure of the random vector.
  Based on this definition, we present a novel approach for modelling  the angular component
  with families of distributions that provide flexibility and can be applied in a moderately large dimensional setting.
  Our focus on the  angular measure is similar to that in \cite{SaNa2014} and \cite{HaCaCh2017}, that consider Bayesian non-parametric approaches. Yet, our approach avoids the problem of dealing with the so called moment constraint by considering a constructive definition of the multivariate GP based on the infinity norm.
  
  The reminder of this paper is outlined as follows. Section~\ref{sec:multivariatepot} comprises a brief review of multivariate PoT, detailing the separation of the radial measure from the angular measure.
  Section~\ref{sec:methodology} details our approach for estimating the angular measure, based on projecting an arbitrary distribution supported in ${\mathbb R}_+^d$ onto unit hyper-spheres defined for $p$-norms. Section~\ref{sec:evaluation} develops criteria for model selection in the support of the angular measure.  Section~\ref{sec:results} explores the efficacy of the proposed approach on a set simulated data, and, acknowledging the relevance of of extreme value theory to climatological events~\citep{jentsch2007,vousdoukas2018,li2019}, estimates the extremal dependence structure for a measure of water vapor flow in the atmosphere, used for identifying atmospheric rivers.  Finally, Section~\ref{sec:conclusion} presents our conclusions and discussion.

Throughout the paper, we adopt the operators $\wedge$ to denote minima, and the $\vee$
  to denote maxima.  Thus $\wedge_i x_i = \min_i x_i$, $\vee_i x_i = \max_i x_i$.  These operators can
  also be applied component-wise between vectors, such as $\bm{a}\wedge\bm{b} = (a_1\wedge b_1, a_2\wedge b_2,\ldots)$.
  
\section{A multivariate PoT model\label{sec:multivariatepot}}
To develop a multivariate PoT model for extreme values consider a $d$-dimensional random vector $\bm{Z} \in {\mathbb R}^d_+ = [0,+\infty)^d$, the positive $d$-dimensional cone. Following the discussion in \cite{goix2017} we assume a form of regular variation, consisting on assuming the existence of a limiting measure $\mu$, such that.
\[
    \lim\limits_{n\to\infty}n\prob{\frac{1}{n}\bm{Z}\geq \bm{z}} = \mu\left([\bm{0},\bm{z}]^c\right),
\]
where the inequality and the interval are taken componentwise. 
In this setting, $\mu$ is the asymptotic distribution of $\bm{Z}$ in extreme regions, and it is referred to as the
  \emph{exponent measure}. Under this condition $\mu$ features the homogeneity property $\mu(tA) = \frac{1}{t}\mu(A)$. 
Define  $R = \max_{i} Z_i$, and assume that $R>1$ with probability one. Also, define $V_i = Z_i/\max_i Z_i$, and assume that $E(V_i)>0$.
These two assumptions together with the homogeneity of $\mu$ correspond to the second condition in Theorem 2.1 of
\cite{ferreira2014}, applied to the finite-dimensional case.
Thus, we have the factorization $\bm{Z} =  R\bm{V}$, where $R$ and $\bm{V}$  are independent. Moreover,
for some measure $\Phi$, $\mu(\cdot)$ is factorized as
\begin{equation}\label{mufact}
    \mu\left( \left\lbrace\bm{Z} : V \in A , R > r\right\rbrace \right) = r^{-1}\Phi(A).
\end{equation}
$R$ describes the radial component, and $\bm{V} \in {\mathbb S}_{\infty}^{d-1}$ the 
angular component of $\bm{Z}$.  Here ${\mathbb S}_\infty^{d-1}$ denotes the positive
orthant of the unit sphere in infinite norm.  $\Phi(\cdot)$ is referred to as the \emph{angular measure}.   
From Equation (\ref{mufact}) we see that $R$ is distributed as a standard Pareto random variable. Conditioning on $R$, we have that
\begin{equation} \label{MGP}
    \prob{\bm{V} \in A \mid R > r} = \frac{r\prob{\bm{V} \in A, R > r}}{r\prob{R > r}}\hspace{0.2cm}
      \xrightarrow[r\to\infty]{~} \hspace{0.2cm} \frac{\Phi(A)}{\Phi({\mathbb S}_{\infty}^{d-1})}.
\end{equation}
 Thus, the angular measure corresponds to the distribution of $\bm{V}$, conditional on $R$ being large.
 As such, it controls the dependence structure of $\bm{Z}$ in the tails. In fact, following \cite{ferreira2014} the distribution function of $\bm{Z}$ is given as
\begin{equation*}
    \prob{Z_1\leq z_1, \ldots ,Z_d\leq z_d} = \expect{\bigvee_{i=1}^d
\frac{V_i}{z_i\wedge 1}} - \expect{\bigvee_{i=1}^d
\frac{V_i}{z_i}}
\end{equation*}
where the expectation is taken with respect to $\Phi$.
In view of this, to obtain a PoT model for $\bm{Z}$, we seek 
a flexible distribution on $\bm{V} \in {\mathbb S}_{\infty}^{d-1}$.

\cite{ferreira2014} refer to the model defined in Equation \eqref{MGP} as a simple 
Pareto process indexed on a finite dimensional set. An alternative approach to 
define such distribution, that provides a link to the multivariate GEV, is discussed 
in Remark 1 of \cite{RoSeWa2018a}. More specifically, 
using the notation in Theorem 6 of \cite{RoSeWa2018a}, define $\bm{S} = \log(\bm{V})$. 
Then, according to Definition 5, $\bm{S}$ is a spectral random vector. Taking 
$E \sim \text{Exp}(1)$, independent of $\bm{V}$, we have that $\bm{Y} = \bm{S} + E$ 
follows a multivariate GP, according to Theorem 6.  Then, using Proposition 4, 
$\bm{Z} = e^{Y}$ is also a multivariate GP. Notice that $R = e^E$ is distributed as 
a standard Pareto random variable.

%\subsection{Pairwise Extremal Dependence\label{subsec:ped}}
A measure that is used to characterize the strength of the dependence,
in the tail, for two random variables $W_1$ and $W_2$, with marginal
distributions $F_1$ and $F_2$ is given by
\[	\chi_{12} = \lim_{u\uparrow 1} \prob{F_1(W_1)>u|F_2(W_2)>u}  \]
\citep{coles2001}.
$\chi_{12}$ provides information about the distribution of extremes for the variable $W_1$
given that $W_2$ is very large.  When $\chi_{12}>0$, $W_1$ and $W_2$ 
are said to
be asymptotically dependent, otherwise they are asymptotically
independent. The following result provides the asymptotic dependence coefficient for two components of $\bm{Z}$. 
\begin{prop}\label{ppchi}
Suppose that $\bm{Z} = R\bm{V}$ with $R\sim Pa(1)$,
$\prob{V_\ell > 0} = 1$ and $\expect{V_\ell}$ exists ,
$\ell=1, \ldots ,d$, then
\begin{equation}
    \label{eqn:chi_ij}
	\chi_{\jmath\ell} = \expect{\frac{V_\jmath}{\expect{V_\jmath}} \wedge \frac{V_\ell}{\expect{V_\ell}}}
\end{equation}
\end{prop}
{\em Proof:}
Denote as $F_\ell$ the marginal distribution of $Z_\ell$. To obtain $\chi_{\jmath\ell}$ we need $Pr(Z_\jmath>z_\jmath,Z_\ell>z_\ell)$, where $z_\ell =
F_\ell^{-1}(u) = \expect{V_\ell}/(1 - u), \;\ell=1,\dots, d$.
Using the fact that
$V_\ell>0, \forall \ell$ almost surely, we have that the former is equal to
\begin{equation*}
\prob{R>\frac{z_\jmath}{V_\jmath}\vee\frac{z_\ell}{V_\ell}}
    = \expect{1\wedge\left(\frac{z_\jmath}{V_\jmath}\vee\frac{z_\ell}{V_\ell}\right)^{-1}}
    = \expect{\frac{V_\jmath}{z_\jmath}\wedge\frac{V_\ell}{z_\ell}}
    = (1 - u)\expect{\frac{V_\jmath}{\expect{V_\jmath}}\wedge\frac{V_\ell}{\expect{V_\ell}}},
\end{equation*}
where the second identity is justified by the fact that $V_i$ is
bounded and $z_i\rightarrow\infty$. The proof is completed by noting
that $\prob{F_i(Z_i)>u} = 1 - u.\hfill\Box$

%\subsection{Multivariate Conditional Survival Function\label{subsec:condsurv}}
For the analysis of extreme values it is of interest to calculate the multivariate conditional survival function. The following result provides the relevant expression, as a function of the angular measure.
\begin{prop}
Assume the same conditions of Proposition \ref{ppchi}. 
Let $\alpha \subset \{1, \ldots ,p\}$ be a collections of indexes. 
Then     
\begin{equation} \label{eqn:condsurv2df}
    \mathrm{Pr}\left[\bigcap_{\ell \in \alpha} Z_\ell > z_\ell \right|\left. \bigcap_{\ell\not\in\alpha} Z_\ell > z_\ell\right] =
      \frac{\expect{\bigwedge_{k = 1}^d 1\wedge \frac{V_k}{z_k}}}{\expect{\bigwedge_{k \not\in\alpha}1\wedge\frac{V_k}{z_k}}}.
  \end{equation}
\end{prop}  
The proof uses a similar approach to the proof of Proposition \ref{ppchi}.

  % EOF
