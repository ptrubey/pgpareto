
\section{Introduction}

The statistical analysis of extreme values focuses on inference for
    rare events that correspond to the tails of probability distributions.
    As such, it is a key ingredient in the risk assessment of
    phenomena that can have strong societal impacts like floods, heat waves,
    high concentration of pollutants, crashes in the financial markets,
    among others. The fundamental challenge of extreme value theory (EVT) is
    to use information, collected over limited periods of time, to
    extrapolate to long time horizons. This sets EVT apart from most of
    statistical inference, where the focus is on the bulk of the
    distribution. Extrapolation to the tails of the distributions is
    possible thanks to theoretical results that give asymptotic descriptions
    of the probability distributions of extreme values. 

Inferential methods for the extreme values of univariate observations
    are well established and software is widely available 
    \cite[see, for example,][]{coles2001}. For variables in one dimension the 
    application of EVT methods considers the asymptotic distribution of either the 
    maxima calculated for regular blocks of data, or the values that exceed a 
    certain threshold. The former leads to a Generalized Extreme Value (GEV) 
    distribution, that depends on three parameters. The latter leads to a Generalized 
    Pareto (GP) distribution, that depends on a shape and a scale parameter. 
    Likelihood-based approaches to inference can be readily implemented in both cases.
    In the multivariate case the GEV theory is well developed 
    \citep[see, for example][]{dehaan2006}, but the inferential problem is  
    complicated by the fact that there is no parametric representation of the GEV. 
    This problem is inherited by the peaks over threshold (PoT) approach and 
    compounded by the  fact that there is no unique definition of an exceedance of 
    a multivariate  threshold, as there is an obvious dependence on the norm that 
    is used to measure  the size of a vector. 

During the last decade or so, much work has been done in the exploration of the 
    definition and properties of an appropriate generalization of the univariate 
    GP distribution to a multivariate setting.  To mention some of the papers on 
    the topic, the work of \citep{rootzen2006} defines the generalized Pareto 
    distribution, with further analysis on these classes of distributions 
    presented in \cite{falk2008} and \cite{michel2008}.  A recent review of the 
    state of the art in multivariate peaks over threshold modelling using 
    generalized Pareto is provided in \cite{rootzen2018} while \cite{RoSeWa2018a} 
    provides insight on the theoretical properties of possible parametrizations. 
    These are use in \cite{KiRoSeWa2019} for likelihood-based models for PoT 
    estimation. A frequently used method for describing the dependence in 
    multivariate distributions is to use a copula. \cite{renard2007}, and 
    \cite{falk2019} provide successful examples of this approach in an EVT 
    framework. 
    
    \cite{ferreira2014} presents a constructive definition of the 
    Pareto process, that generalizes the GP to an infinite dimensional setting. It 
    consists of decomposing the process into independent radial and angular 
    components. Such an approach can be used in the finite dimensional case, where 
    the angular component contains the information pertaining to the dependence 
    structure of the random vector. Based on this definition, we present a novel 
    approach for modelling  the angular component with families of distributions 
    that provide flexibility and can be applied in a moderately large dimensional 
    setting.  Our focus on the angular measure is similar to that in \cite{boldi2007},
    \cite{SaNa2014} and \cite{HaCaCh2017}, that consider Bayesian non-parametric 
    approaches. Yet, our approach differs in that it is established in the 
    peaks-over-threshold regime, and avoids the problem of dealing with the so 
    called moment constraint by considering a constructive definition of the 
    multivariate GP, based on the infinity norm. The approach proposed in this paper
    adds to the growing literature on Bayesian models for multivariate extreme
    value analysis (see, for example, \cite{boldi2007},\cite{guillotte2011},
    \cite{SaNa2014},\cite{hanson2017}), providing a model that has strong computational
    advantages due its structural simplicitly, achieves flexibility using 
    a mixture model, imposes no moment constraints, and scales well to moderately 
    large dimensions.
  
The remainder of this paper is outlined as follows. 
    Section~\ref{sec:multivariatepot} comprises a brief review of multivariate PoT, 
      detailing the separation of the radial measure from the angular measure.
    Section~\ref{sec:methodology} details our approach for estimating the angular 
      measure, based on projecting an arbitrary distribution supported in ${\mathbb R}_+^d$ 
      onto unit hyper-spheres defined by $p$-norms. 
    Section~\ref{sec:evaluation} develops criteria for model selection in the support 
      of the angular measure.  
    Section~\ref{sec:results} explores the efficacy of the proposed approach on a set 
      of simulated data, and, acknowledging the relevance of extreme value theory to 
      climatological events~\citep{jentsch2007,vousdoukas2018,li2019}, estimates the 
      extremal dependence structure for a measure of water vapor flow in the atmosphere, 
      used for identifying atmospheric rivers.  
    Finally, Section~\ref{sec:conclusion} presents our conclusions and discussion.

Throughout the paper, we adopt the operators $\wedge$ to denote minima, and 
    the $\vee$ to denote maxima.  Thus $\wedge_i s_i = \min_i s_i$, and 
    $\vee_i s_i = \max_i s_i$.  These operators can also be applied component-wise 
    between vectors, such as $\bm{a}\wedge\bm{b} = (a_1\wedge b_1, a_2\wedge b_2,\ldots)$.  
    We use uppercase to indicate random variables, lowercase to indicate points, and
    bold-face to indicate vectors or matrices thereof.
  
\section{A multivariate PoT model\label{sec:multivariatepot}}
To develop a multivariate PoT model for extreme values consider a 
    $d$-dimensional random vector $\bm{W} = (W_1, \ldots ,W_d)$ with
    cumulative distribution $F$. Following \cite{RoSeWa2018a}, assume 
    that there exists sequences of vectors $\bm{a}_n$ and $\bm{b}_n$,
    and a $d$-variate distribution $G$ such that 
    $\lim_{n\rightarrow\infty} F^n(\bm{a}_n \bm{w} + \bm{b}_n) = 
    G(\bm{w})$. $G$ is a $d$-variate generalized extreme value 
    distribution. It follows that
    \[
        \lim\limits_{n\rightarrow\infty} \prob{\bm{a}^{-1}_n (\bm{W} - \bm{b}_n) 
      \leq \bm{w} \mid \bm{W} \not\leq \bm{b}_n)} \\ 
        = \frac{\log G(\bm{w} \wedge \bm{0}) 
        - \log G(\bm{w}) }{\log G(\bm{0})} = H(\bm{w}).
    \]
    $H$ is a multivariate Pareto distribution, and corresponds to a joint
    distribution conditional on exceeding a multivariate threshold. In addition to
    a copula function, $H$ depends on two $d$-dimensional vectors of 
    parameters that we denote as $\bm{\xi}$ for the shapes and $\bm{\sigma}$ 
    for the scales. \cite{RoSeWa2018a} provide a number of stochastic 
    representations for $H$.  In particular, denote as $Z$ a
    random variable with distribution $H$ where $\bm{\xi}= 1$ and 
    $\bm{\sigma} = 0$.  Then, Remark 1 justifies the representation in 
    \cite{ferreira2014}, giving $\bm{Z} = \bm{V}R$
    where $R$ and $\bm{V}$ are independent. $R = \|\bm{Z}\|_\infty$ is 
    distributed as a standard Pareto random variable, and $\bm{V} = 
    \bm{Z}/\|\bm{Z}\|_\infty$ is a random vector in  
    $\mathbb{S}_{\infty}^{d-1}$, the positive orthant of the 
    unit sphere under $\mathcal{L}_{\infty}$ norm, with distribution $\Phi$. 
    This representation is central to the methods proposed in this paper.
    $R$ and $\bm{V}$ are referred to, respectively, as the {\em radial} 
    and {\em angular} components of $H$. The angular measure controls 
    the dependence  structure of $\bm{Z}$  in the tails. In view of 
    this, to obtain a  PoT model we seek a flexible model for the 
    distribution of $\bm{V} \in {\mathbb S}_{\infty}^{d-1}$.

The approach considered in \cite{rootzen2018} focuses on the limiting
    conditional distribution $H$. An alternative approach consists of assuming
    that regular variation \cite[see, for example,][]{resnick2008extreme} holds 
    for the limiting distribution of $\bm{W}$, implying
    \[
        \lim\limits_{n\rightarrow\infty} n \prob{n^{-1} \bm{W}\in A} = 
        \mu(A),
    \]
    where $\mu$ is termed the exponent measure. $\mu$ has the homogeneity property
    $\mu(tA) = t^{-1}\mu(A)$. Letting $\rho = \|W\|_p, p>0$ 
    and $\bm{\theta} = \bm{W}/\rho$, define  $\Psi(B) = \mu(\{\bm{w} : \rho>1, 
    \bm{\theta} \in B\})$, which is referred to as the angular measure. After 
    some manipulations, we obtain that
    \begin{equation}
    \label{exponent-mu}
        \lim\limits_{r\rightarrow\infty} 
        \text{Pr}\left[\bm{\theta}\in A \rvert \rho>r\right] = 
            \frac{\Psi(A)}{\Psi({\mathbb S}_p^{d-1})} .
    \end{equation}
    Thus, a model for the exponent measure induces a model for the limiting distribution
    conditional on the observations being above a threshold defined with respect
    to their $p$-norm. The constraint that all marginals of $\mu$ correspond to a
    standard Pareto distribution leads to the so called moment constraints on $\Psi$. 
    Inference for the limiting distribution of the exceedances
    needs to account for the normalizing constant in Equation~\eqref{exponent-mu}, as well
    the moment constraints. 
    
    % In addition, the moment constraints can impose unrealistic
    % symmetries in the data distribution.

% EOF
